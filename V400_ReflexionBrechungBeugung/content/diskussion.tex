\section{Diskussion}
\label{sec:Diskussion}

Wie zu erwarten, liegt die Steigung der linearen Regression der Ausfallswinkel in Aufgabenteil 1 bei 
recht genau 1. Die Abweichungen haben größtenteils systematische Ursachen. Da die Werte stets analog 
anhand einer Unterlage abgemessen wurden, sind Fehler beim Ablesen sehr wahrscheinlich.
Nicht zuletzt, weil man durch ein Glas hindurch auf die Unterlage schaut, so dass auch beim Ablesen
der Brechungseffekt sehr gut erkennbar ist. Hinzu kommt außerdem bei der reflektierenden 
Platte, dass sie mit großer Wahrscheinlichkeit nicht perfekt glatt war. Dadurch wird das 
Licht nicht einwandfrei zurückgeworfen und es kommt zu Abweichungen im Ausfallswinkel. \\
 
Kronglas besitzt nach chemie.de einen Brechungsindex von 1.5-1.6, sodass der hier bestimmte Wert von 
1.53 gut ins Bild passt. Bei der Planparallelen Platte können mögliche Verunreinigungen des 
Plexiglases beispielsweise durch Fingerabdrücke oder Kratzer für Fehler im Brechungswinkel oder 
Strahlversatz führen. \\

Die vom Hersteller angegebene Wellenlänge liegt mit 635 nm deutlich über der durch den Versuch 
bestimmten von 328 nm. Auch liegen zwischen den einzelnen Messpunkten keine großen Abweichungen 
vor, so dass es sich vermutlich um statistische Messfehler handelt. Beim Versuchsaufbau ist 
zu bedenken, dass das Papier mit Winkelskala von Hand aufgestellt wurde, also keinen perfekten 
Bogen beschrieb. Da die Richtigkeit der abgelesenen Werte auch davon abhing, ob die Versuchsanordnung
korrekt auf die Unterlage ausgerichtet war, kommen als weitere Fehlerquellen ein Verrutschen
der Unterlage oder der Versuchsanordnung hinzu, die zu Abweichungen beizutragen.\\
%noch prozentuale Abweichungen bestimmen?
