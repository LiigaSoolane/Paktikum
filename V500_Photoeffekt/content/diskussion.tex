\section{Diskussion}
\label{sec:Diskussion}

Der aus der linearen Regression bestimmte Wert für die Plancksche Konstante lautet
$h=4.46 \cdot 10^{-15} eVs$
Von dem Literaturwert $h = 4.136 \cdot 10^{-15}$ Vs e weicht er also um 
$\Delta h= 7.8\% $ ab. Trotz der Tatsache, dass nur vier Werte zur Bestimmung von 
h verwendet wurden, ist diese Abweichung sehr gering. 
Bei der Versuchsdurchführung ist aufgefallen, dass die Nadel des Pikoamperemeters 
ohne Veränderung der zu untersuchenden Parameter schwankt. Außerdem war der Raum nicht komplett 
abgedunkelt, so dass auch anderes Licht auf die Photokathode fällt.

