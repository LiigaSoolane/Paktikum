\section{Diskussion}
\label{sec:Diskussion}

\subsection{Einzelspalt}
    Der Hersteller gibt für den Einzelspalt eine Breite von 15 mm an. In der Auswertung wurde 
    die Spaltbreite jedoch zu 0.157 mm bestimmt. Somit liegt eine Abweichung um 4.9 \% vor.
    
\subsection{Doppelspalt}
    Die Herstellerangaben zum Doppelspalt lauten:
    \begin{align*}
        Breite = 0.15 mm\\
        Abstand = 0.5 mm\\
        s = 0.65 mm\\
    \end{align*}
    Diese weichen von den tatsächlich bestimmten Werten 
    \begin{align*}
        Breite = 0.138 mm\\
        s = 0.498 mm\\
    \end{align*}
    ein wenig ab (Die Abweichung der Breite des Spaltes wurde bestimmt zu 8.19\% bestimmt.
    Es könnte sein, dass 
    der Spalt während der Messungen verutscht ist und somit die Messergebnisse verfälscht wurden.
    Außerdem konnte das Amperemeter nicht auf die niedrigste Stufe eingestellt werden, weshalb
    exakte Messungen erschwert wurden. Des weiteren kommen als Fehlerfaktoren auch noch der 
    Dunkelstrom und Restlicht, verursacht durch eine offenstehende Tür und eine Steckdosenleiste, 
    ins Spiel.
    Außerdem gibt es erhebliche Defizite beim Curvefit. Dadurch dass die Extrema im Vergleich zu den 
    Messpunkten eng beieinander liegen, wird das fitten erschwert und es müssen Parameter vorgegeben 
    werden, an denen sich der Fit orientieren soll.

