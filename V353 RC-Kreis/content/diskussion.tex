\section{Diskussion}
\label{sec:Diskussion}
Die zu berechnenden Werte für RC sind mit
\begin{equation*}
    RC_1 = 0.000931 \pm 0.000032s
\end{equation*}
\begin{equation*}
    RC_2 = 0.007222 \pm 0.002084s
\end{equation*}
\begin{equation*}
    RC_3 = 0.000680 \pm 0.000052s
\end{equation*}
um bis zu einer Größenordnung verschieden. Auffällig ist, dass der durch die frequenzabhängigen Amplituden bestimmte Wert $RC_2$ stärker von den beiden 
anderen Werten abweicht. $RC_2$ ist 7,76 mal so groß wie $RC_1$ und 10,62 mal so groß wie $RC_3$. Wohingegen $RC_1$ lediglich 36,91\% größer als 
$RC_3$ ist, also 1,3691 mal so groß ist. Dies ist zum einen durch Messungenauigkeiten zu erklären, da Werte an einem Oszilloskop Ablesen relativ ungenau ist. 
Zum anderen sind die Messpunkte nicht optimal gewählt, wodurch die Ausgleichsrechnungen nicht optimal sind. Anhand der Graphen sieht man, dass 
die Ausgleichskurven im bereich der kleinen Frequenzen ,vorallem bei Abbildung \ref{fig:c}, stärker von den Messwerten abweicht als bei größeren Frequenzen. 
Hier hätten mehr Messpunkte bei geringeren Frequenzen gewählt werden müssen, um bessere Ausgleichskurven zu erhalten.
Die Phasenverschiebung, in Abbildung \ref{fig:d} dargestellt, bewegt sich wie erwartet zwischen 0 und $\pi/2$, sowohl bei derm Theoretischen Praphen als 
auch bei den eingetragenen Messwerten. Zudem stimmen die Messwerte und die Kurve nahezu überein. Allerdings sieht man auch hier wieder, dass die Messwerte nicht 
optimal gewählt wurden. Dies ist unteranderem dem geschuldet, dass bei 20Hz auf dem Oszilloskop kein Bild angezeigt wurde, welches ein Ablesen von Werten 
möglich gemacht hätte.