\section{Diskussion}
\label{sec:Diskussion}

\subsection{Statische Methode}
Die wichtigsten Messergebnisse sind in \ref{tab:pieceofshit} zusammengefasst.

\begin{table}
    \centering
    \caption{Ergebnisse der Messung mit statischer Erhitzung.}
    \label{tab:pieceofshit}
    \begin{tabular}{c c c c c}
        \toprule
        Messwert & Messing, breit & Messing, schmal & Aluminium & Edelstahl \\
        \midrule
        Temperatur nach 700s & $316.41 \si{\kelvin}$  & $309.56 \si{\kelvin} $ & $320.26 \si{\kelvin}$ & $305.83 \si{\kelvin}$\\
        Wärmestrom nach 300s & $0.645 \si{\kelvin\per\s}$ & $0.316 \si{\kelvin\per\s}$ & $0.169 \si{\kelvin\per\s}$ & $0.361 \si{\kelvin\per\s}$\\
        Wärmestrom nach 900s & $0.497 \si{\kelvin\per\s}$ & $0.264 \si{\kelvin\per\s}$ & $0.045 \si{\kelvin\per\s}$ & $0.298 \si{\kelvin\per\s}$\\
        \bottomrule
    \end{tabular}
\end{table}% patetica, eroica

\noindent Anhand dieser Werte ist zu erkennen, das Aluminium der beste Wärmeleiter ist, während Edelstahl am wenigsten gut leitet. Das ist genau das Ergebnis, das zu erwarten war.
Auch die Temperaturverläufe (Abbildungen \ref{fig:messingstat} und \ref{fig:aluedelstat}) 
spiegeln dieses Ergebnis sehr gut wieder; es ist zu beobachten, dass Aluminium sich schneller an einen Grenzwert annähert als Edelstahl und dass es 
insgesamt eine höhere Temperatur erreicht. 

\subsection{Dynamische Methode}
Tabelle \ref{tab:vergleich} zeigt die aus den Messwerten berechneten Wärmeleitfähigkeiten im Vergleich mit den Theoriewerten (Quelle \cite{litval}).
\begin{table}
    \centering
    \caption{Vergleich der Mess- und Theoriewerte der Wärmeleitfähigkeiten.}
    \label{tab:vergleich}
    \begin{tabular}{c c c c}
        \toprule
        Metall & Messwert$\mathbin{/} \si{\watt\per\m\kelvin}$ & Theoriewert$\mathbin{/} \si{\watt\per\m\kelvin}$ & Abweichung in Prozent \\
        \midrule
        Messing & $100 \pm 60 $ & 120 & $10 \pm 50$\\
        Aluminium & $250 \pm 240$ & 235 & $10 \pm 100$\\
        Edelstahl & $19 \pm 16$& 21 & $10 \pm 80$\\
        \bottomrule
    \end{tabular}
\end{table}

\noindent Es fällt auf, dass die Unsicherheiten in den berechneten Werten sehr hoch sind, sodass sowohl Abweichungen von wenigen Prozent aber auch 100$\%$-ige Abweichungen möglich sind.\\
Da nur wenige Perioden gemessen wurden, kommt ein großer statistischer Fehler zum Tragen; zusätzlich dazu gibt es beim manuellen Umschalten von Kühlen auf Erhitzen, das bestenfalls immer nach
genau einer halben Periodendauer erfolgen sollte, ein hohes Potential für menschlichen Fehler. Wenn die Periodendauer während der Messung um einige Sekunden variiert, führt das zu einer Schwankung
in den Messwerten. Zudem werden die Amplituden und Phasendifferenzen zwar teilweise durch Scientific Python ausgelesen, da das aber bei der Messung von Edelstahl nicht möglich 
war (aufgrund der geringen Ausprägung der Amplituden), musste hier anhand des Plots abgelesen werden. Das führt auch zu weiteren Unsicherheiten. \\%nessuna razza
\\

Tabelle \ref{tab:die} zeigt die Werte für die Wellenlängen der verschiedenen Temperaturwellen bei periodischer Anregung.

\begin{table}
    \centering
    \caption{Wellenlängen der Temperaturwellen.}
    \label{tab:die}
    \begin{tabular}{c c c c}
        \toprule
        Metall & $\lambda \mathbin{/} \si{\m}$ & $\lambda_{\text{theoretisch}} \mathbin{/} \si{\m}$ & Abweichung in Prozent \\
        \midrule
        Messing & $0.18 \pm 0.06$ & 0.191 & $7 \pm 29$\\
        Aluminium & $0.33 \pm 0.16$ & 0.318 & $0 \pm 50$\\
        Edelstahl & $0.12 \pm 0.05$ & 0.128 & $10 \pm 40$\\
        \bottomrule
    \end{tabular}
\end{table}

Eindeutig haben die Schwankungen bei Aluminium die größte Wellenlänge, während die bei Edelstahl am kleinsten ist. Wie zuvor sind die Schwankungen sowohl in den Messwerten als auch den Abweichungen
sehr groß. Mögliche Ursachen davon wurden bereits diskutiert.