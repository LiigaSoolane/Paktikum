%\section{Durchführung}
%\label{sec:Durchführung}
%
%\subsection{Aufbau}
%\begin{figure}
% \centering
% \includegraphics[width=textwidth]{}
% \label{fig:aufbauversuch}
% \caption{Skizze der Messapparatur.}
%\end{figure}
%
%\noindent Zur Durchführung des Versuchs wurde die Versuchsanordnung verwendet,
%die Abbildung \ref{fig:aufbauversuch} zeigt. Vom Anodendraht abfließende Ladung
%erzeugt am Widerstand einen Spannungsimpuls, der über den Kondensator C ausgekoppelt wird,
%im Verstärker vergrößert und im Zählgerät registriert, oder auf dem Schirm
%eines Ozilloskops dargestellt.\\
%
%\subsection{Charakteristik}
%Um die Charakteristik des Zählrohrs aufzunehmen, wird eine Elektronenquelle vor dem
%Fenster platziert. Hier wurde eine $^204Tl$-Quelle verwendet. Bei der Platzierung
%wurde dabei darauf geachtet, dass bei mittlerer Spannung eine Rate von
%100 Impulsen pro Sekunde nicht überschritten wurde. Dies ist erforderlich,
%um Totzeit-Korrekturen zu vermeiden. \\
%In 10-Volt-Schritten wurde die Anzahl an Zerfällen pro Zeitintervall gemessen, alle 
%50 Volt wurde zusätzlich der Strom am Amperemeter abgelesen.\\
%Die Messzeit von $t = 60\si{\s}$ wurde gewählt, um den statistischen Fehler
%auf unter 1\% abzusenken, da die Empfindlichkeit dieser Messung ausgesprochen
%hoch ist. 
%
%\subsection{Totzeit}
%Einmal wurde die Totzeit mithilfe des Oszilloskops bestimmt, dazu wurde 
%die Zeitachse auf auf $100 \si{\micro\s}$ eingestellt.\\
%Zudem wurde die Zwei-Quellen-Methode genutzt. Dazu wurde eine weitere
%$^204 Tl$-Quelle verwendet, die, um eine andere Impulsrate zu erzielen,
%näher am Zählrohr platziert wurde. 