\section{Diskussion}
\label{sec:Diskussion}

Sowohl für die Wheatstonesche Brücke als auch für die Kapazitätsmessbrücke weichen 
die durch den Versuch bestimmten Werte kaum von denen vom Hersteller angegebenen Werten
ab. Das lässt darauf schließen, dass die Messinstrumente und anderen Bauteile 
exakt arbeiten und den Herstellerangaben entstprechen.\\
Bei der Induktivitätsmessbrücke ist zu erwarten, dass die Werte ein wenig abweichen, da
Spulen nicht als Vergleichselement geeignet sind, da durch den Innernwiderstand der Spule
thermische Energie verloren geht. Tatsächlich wurde beobachtet, dass
die selbe Induktivität mit einer Maxwellbrücke wesentlich genauer gemessen wurde.\\
Wie in Abb. \ref{fig:plot} dargestellt, weicht die Theoriekurve für hohe Frequenzen
von den Messwerten ab. Inbesondere die Genauigkeit um $v_0$ lässt auf den tatsächlich recht
geringer Klirrfaktor schließen. 