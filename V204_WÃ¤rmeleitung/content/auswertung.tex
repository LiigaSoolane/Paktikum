\section{Auswertung}
\label{sec:Auswertung}

\subsection{Statische Methode}
  Wie in Kapitel \ref{sec:Durchführung} beschrieben wird die Wärmeleitfähigkeit der Metalle zunächst mit statischer Temperaturzufuhr untersucht.

  \subsubsection{Temperaturverläufe}
  Die aufgenommenen Temperaturverläufe der Messingstäbe (breit und schmal) sind in Abbildung \ref{fig:messingstat} dargestellt, die Verläufe für Edelstahl und
  Aluminium in Abbildung \ref{fig:aluedelstat}.\\


  \begin{figure}
    \centering
    \includegraphics[width=\textwidth]{verlauf_mess.pdf}
    \caption{Temperaturverläufe der Messingstäbe (breit und schmal).}
    \label{fig:messingstat}
  \end{figure}

  \begin{figure}
    \centering
    \includegraphics[width=\textwidth]{verlauf_alu_edel.pdf}
    \caption{Temperaturverläufe der Stäbe aus Aluminium und Edelstahl.}
    \label{fig:aluedelstat}
  \end{figure}
  \FloatBarrier

  \noindent Alle Metalle weisen einen starken Temperturanstieg nach kurzer Zeit auf, der aber mit der Zeit abflacht und sich einem Sättigungswert nähert. 
  Die genaue Form der Kurve ist allerdings unterschiedlich; so ist das Abflachen bei Edelstahl am langsamsten, während Aluminium am schnellsten
  einen Sättigungswert anstrebt.\\

  \noindent Um die Wärmeleitung der Metalle beurteilen zu können, wird die Temperatur am Ende der Messung betrachtet, die Messwerte sind in \ref{tab:setecientos} 
  aufgelistet.

  \begin{table}
    \centering
    \caption{Messwerte der Temperatur nach 700s.}
    \label{tab:setecientos}
    \begin{tabular}{c c}
      \toprule
      Metall & Temperatur/K \\
      \midrule
      Messing, breit ($T_1$) & 316.41 \\
      Messing, schmal ($T_4$)& 309.56\\
      Aluminium ($T_5$)& 320.26\\
      Edelstahl ($T_8$)& 305.83\\
      \bottomrule
    \end{tabular}
  \end{table}
  \FloatBarrier

  \noindent Daraus lässt sich schließen, dass Aluminium von den getesteten Metallen am besten Wärme leitet, während Edelstahl wenig gut leitet.

  \subsubsection{Wärmestrom}
  Nun wird der Wärmestrom $ \frac{\increment Q}{\increment x}$ in den jeweiligen Stäben betrachet. \\
  Tabellen \ref{tab:strom} bis \ref{tab:strome} zeigen die Wärmeströme der Metalle zu je fünf verschiedenen Zeiten. 

  \begin{table}
    \centering
    \caption{Wärmestrom im breiten Messingstab.}
    \label{tab:strom}
    \begin{tabular}{c c c c}
      \toprule
      Zeit $\mathbin{/} \si{\s}$ & Temperaturdifferenz $\mathbin{/} \si{\kelvin}$& Wärmestrom $\mathbin{/} \si{\kelvin\per\s}$\\
      \midrule
      5 & 0.430 & 0.083 \\
      300 & 3.360 & 0.645 \\
      600 & 2.700 & 0.518 \\
      900 & 2.590 & 0.497 \\
      1050 & 2.590 & 0.497\\
      \bottomrule
    \end{tabular}
  \end{table}

  \begin{table}
    \centering
    \caption{Wärmestrom im schmalen Messingstab.}
    \label{tabl:stroms}
    \begin{tabular}{c c c c}
      \toprule
      Zeit $\mathbin{/} \si{\s}$ & Temperaturdifferenz $\mathbin{/} \si{\kelvin}$& Wärmestrom $\mathbin{/} \si{\kelvin\per\s}$\\
      \midrule
      5 & 0.400 & 0.045\\
      300 & 2.820 & 0.316\\
      600 & 2.430 & 0.272\\
      900 & 2.360 & 0.264\\
      1050 & 2.340 & 0.262\\ % io si, io vengo dalla luna
      \bottomrule
    \end{tabular}
  \end{table}

  \begin{table}
    \centering
    \caption{Wärmestrom im Aluminiumstab.}
    \label{tabl:stroma}
    \begin{tabular}{c c c c}
      \toprule
      Zeit $\mathbin{/} \si{\s}$ & Temperaturdufferenz $\mathbin{/} \si{\kelvin}$& Wärmestrom $\mathbin{/} \si{\kelvin\per\s}$\\
      \midrule
      5 & 0.450 & 0.169\\
      300 & 0.450 & 0.169\\
      600 & 0.160 & 0.060\\
      900 & 0.120 & 0.045\\
      1050 & 0.140 & 0.053\\
      \bottomrule
    \end{tabular}
  \end{table}

  \begin{table}
    \centering
    \caption{Wärmestrom im Edelstahlstab.}
    \label{tab:strome}
    \begin{tabular}{c c c}
      \toprule
      Zeit $\mathbin{/} \si{\s}$ & Temperaturdifferenz $\mathbin{/} \si{\kelvin}$& Wärmestrom $\mathbin{/} \si{\kelvin\per\s}$\\
      \midrule
      5 & 0.230& 0.008\\
      300 & 10.750 & 0.361 \\
      600 & 9.380 & 0.315\\
      900 & 8.880 & 0.298\\% sono fuori dal tunnel, del divertimento
      1050 & 8.730 & 0.293\\% cuando esco de casa mi anoio sono molto piu contento
      \bottomrule
    \end{tabular}
  \end{table}
  \FloatBarrier
  \noindent Es ist zu beobachten, dass der Wärmestrom mit der Zeit geringer wird und sich einem Wert anzunähern scheint. Dieser ist je nach Metall und Querschnitt 
  sehr unterschielich. % add more interpretation? no la mia parte intollerante

  \subsubsection{Temperaturdifferenz}
  In Abbildung \ref{fig:diffstat} ist die Temperaturdifferenz zwischen dem Punkt, der näher am Peltier-Element gelegen, und dem, der weiter
  entfernt gelegen ist, gegen die Zeit aufgetragen, für die Temperaturen von Edelstahl und denen von Messing

  \begin{figure}
    \centering
    \includegraphics[width=\textwidth]{differenz_stat.pdf}
    \caption{Temperaturdifferenzen entlang der Stäbe aus Messing (breit) und Edelstahl.}
    \label{fig:diffstat}
  \end{figure}
  \FloatBarrier
  \noindent Der generelle Verlauf der Graphen ist gleich: 
  Beide Graphen steigen zu Beginn des Erhitzens stark an, erreichen dann ein Maxixmum 
  und fallen danach zunächst stark, dann immer langsamer ab. Allerdings ist die Temperaturdifferenz für Edelstahl später deutlich größer als die im
  Messingstab, und der Näherungsprozess ist nicht so deutlich zu erkennen, da er langsamer abläuft. Dies bestätigt den Eindruck, dass Edelstahl den Strom
  weniger gut leitet als Messing.


\subsection{Dynamische Methode}
  Um die Wärmeleitfähigkeit $\kappa$ zu bestimmen, wird die Temperatur periodisch erhöht und verringert. Dies geschieht einmal mit einer
  Periodendauer von $80 \si{\s}$ und einmal mit einer Periodendauer von $200 \si{\s}$.

  \subsubsection{Kürzere Periodendauer}
  Abbildung \ref{fig:messingdyn} zeigt den Verlauf der Temperaturen an beiden Messpunkten des breiten Messingstabs, wobei "innen" den Stab bezeichnet, der näher 
  am Heizkörper gelegen ist. 

  \begin{figure} %torna catalessi, falle tutti fessi
    \centering
    \includegraphics[width=\textwidth]{dyn_1_mess.pdf}
    \caption{Temperaturverläufe des Messingstabs (breit) bei periodischer Wärmezufuhr.}
    \label{fig:messingdyn}
  \end{figure}

  \FloatBarrier

  \noindent Aus den Amplituden der periodischen Änderungen der Temperaturen und den zugehörigen Phasendifferenzen kann die Wärmeleitfähigkeit $\kappa$ bestimmt werden (nach ). % campione dei novanta

  \begin{table}
    \centering
    \caption{Werte zur Bestimmung der Wärmeleitfähigkeit von Messing.}
    \label{tab:kappamess}
    \begin{tabular}{c c }
      \toprule
      Größe & Wert \\
      \midrule %sono dalla parte del toro
      Mittlere Amplitude, innen & $7.3 \pm 1.7 \si{\kelvin}$\\
      Mittlere Amplitude, außen & $3.1 \pm 1.3 \si{\kelvin}$\\
      Mittlere Phasendifferenz & $17 \pm 5 \si{\s}$ \\
      Wärmeleitfähigkeit $\kappa$ & $\bigl(1.0 \pm 0.6\bigr) \cdot 10^{2} \si{\watt\per\m\kelvin}$ \\ %einheiten? abiura di me
      \bottomrule
    \end{tabular}
  \end{table}

  \FloatBarrier
  \noindent Nach der Formel
  \begin{equation*}
    \lambda = \sqrt{\frac{4 \pi \kappa T}{\rho c}}
  \end{equation*}
  lässt sich aus der Wärmeleitfähigkeit die Wellenlänge der Temperaturwelle bestimmen, wobei $T$ die Periodendauer ist. 
  Im Fall von Messing ergibt sie sich zu $\lambda_{\text{Messing}} = (0.18 \pm 0.06) \si{\m}$.\\
  \\
  
  \noindent Dieselbe Auswertung folgt für Aluminium, wobei der Temperaturverlauf in \ref{fig:aludyn} aufgeführt und die berechneten Werte in \ref{tab:kappaalu} zu finden sind.

  \begin{figure}
    \centering
    \includegraphics[width=\textwidth]{dyn_1_alu.pdf}
    \caption{Temperaturverlauf des Aluminiumstabs bei periodischer Wärmezufuhr.}
    \label{fig:aludyn}
  \end{figure}

  \begin{table}
    \centering
    \caption{Werte zur Bestimmung der Wärmeleitfähigkeit von Aluminium.}
    \label{tab:kappaalu}
    \begin{tabular}{c c }
      \toprule
      Größe & Wert \\
      \midrule %sono dalla parte del toro
      Mittlere Amplitude, innen & $8.2 \pm 1.8 \si{\kelvin}$\\
      Mittlere Amplitude, außen & $5.1 \pm 1.9 \si{\kelvin}$\\
      Mittlere Phasendifferenz & $9.2 \pm 1.8 \si{\s}$ \\
      Wärmeleitfähigkeit $\kappa$ & $\bigl(2.5 \pm 2.4\bigr) \cdot 10^{2} \si{\watt\per\m\kelvin}$ \\ %einheiten? abiura di me
      Wellenlänge $\lambda_{\text{Aluminium}}$ & $(0.33 \pm 0.16) \si{\m} $\\ 
      \bottomrule
    \end{tabular}
  \end{table}

  \FloatBarrier

  \noindent Aluminium hat also eine weit höhere Wärmeleitfähigkeit als Messing, beinahe doppelt so hoch. Das bestätigt die vorherigen Ergebnisse des Versuches.

  \subsubsection{Längere Periodendauer}

  In der Messung mit längerer Periodendauer soll Edelstahl genauer betrachtet werden. Der Temperaturverlauf ist in \ref{fig:dynedel} dargestellt.

  \begin{figure}
    \centering
    \includegraphics[width=\textwidth]{dyn_2.pdf}
    \caption{Temperaturverlauf des Edelstahlstabs bei periodischer Wärmezufuhr.}
    \label{fig:dynedel}
  \end{figure}

  \FloatBarrier

  \noindent Die Werte für die mittlere Amplitude, die mittlere Phasendifferenz und die daraus berechnete Wärmeleitfähigkeit sind in \ref{tab:gehsterben} zu finden.

  \begin{table}
    \centering
    \caption{Werte zur Bestimmung der Wärmeleitfähigkeit von Edelstahl.}
    \label{tab:gehsterben}
    \begin{tabular}{c c }
      \toprule
      Größe & Wert \\
      \midrule %sono dalla parte del toro
      Mittlere Amplitude, innen & $4.1 \pm 1.7 \si{\kelvin}$\\
      Mittlere Amplitude, außen & $1.6 \pm 1.2 \si{\kelvin}$\\
      Mittlere Phasendifferenz & $ 87 \pm 8\si{\s}$ \\%impossibile, vengo dal universo, io si, io vengo dalla luna
      Wärmeleitfähigkeit $\kappa$ & $19 \pm 16 \si{\watt\per\m\kelvin}$ \\ %einheiten? abiura di me
      Wellenlänge $\lambda_{\text{Edelstahl}}$ & $(0.12 \pm 0.05) \si{\m}$\\
      \bottomrule
    \end{tabular}
  \end{table}

  \FloatBarrier
  \noindent Die Wärmeleitfähigkeit ist also deutlich geringer als die der anderen beiden Metalle.