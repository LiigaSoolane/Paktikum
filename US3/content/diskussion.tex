\section{Diskussion}
\label{sec:Diskussion}

Bei den berechneten Strömungsgeschwindigkeiten fällt auf, dass sie für gleiche Flussgeschwindigkeiten stark von den Innendurchmessern der 
verwendeten Röhren abhängen. Wie zu erwarten sind die Strömungsgeschwindigkeiten beim schmalsten Rohr mit 7mm Innendurchmesser immer am größten, 
danach kommen die des mittleren Rohres und am geringsten sind die Strömungsdurchmesser beim dicksten Rohr mit 16mm Innendurchmesser.
Ebenso ist zusehen, dass bei gleichem Rohr und gleicher Flussgeschwindigkeit die Strömungsgeschwindigkeiten in der selben größenordnung sind. 
Daraus lässt sich schließen, dass die Strömungsgeschwindigkeit nicht vom Dopplerwinkel abhängt, was zu erwarten war. Die bestimmten Geschwindigkeiten 
beim Prismawinkel von 30$^{\circ}$ sind alle negativ, da bei diesem Prismawinkel die Ultraschallsonde in die entgegengesetzte Fließrichtung 
der Dopplerphantomflüssigkeit zeigt. Das Vorzeichen kann daher ignoriert werden. 
Bei Graphik \ref{fig:KeineAhnung} fällt auf, dass das Verhältnis der Frequenzverschiebung zum $cos(\alpha)$ des Dopplerwinkels linear mit steigender 
Strömungsgeschwindigkeit ansteigt. Dies hängt nicht vom Durchmesser des Rohres ab. Dies zeigt die Abhängigkeit der Frequenzverschiebung vom 
Dopplerwikel $\alpha$. 
Bei der Bestimmung des Strömungsprofiels und somit bei Graphik \ref{fig:70P} und \ref{fig:45P}, ist zu sehen wie sowohl die Strömungsgeschwindigkeit als auch 
die Intensität zur mitte des Rohres hin anwächst und an den Seiten abfällt. Bis auf die beiden Intensitätspunkte bei $15\mu s$ und $16\mu s$, bei 
einer 70\%-igen Flussgeschwindigkeit, ist dies eindeutig. Diese beiden Fehler könnten durch ungenaues Messen zustande gekommen sein, eine 
naheliegende Erklärung für eine geringer gemessene Intensität sind Luftblasen in dem Gel zwischen Ultraschallsonde und Prisma oder 
zwischen Prisma und Schlauch. Diese würden die Ultraschallblasen z.B. reflektieren, wenn diese von der Flüssigkeit zurück zur Sonde laufen, 
wodurch diese nicht gemessen werden können und die gemessene Intensität sinkt.
Die restlichen Messpunkte zeigen, dass die Strömungsgeschwindigkeit am Rand geringer ist. Die ebenfalls am Rand geringere Intensität 
gibt einen Hinweis auf Wirbel am Rand. Durch diese würde sowohl die Strömingsgeschwindigkeit sinken, als auch die Intensität, da 
die Ultraschallwellen an diesen Wirbeln gestreut werden. 
Insgesammt sind in allen drei Aufgabenteilen die zu erwartenden Phänomene beobachtet worden, was auf gute Messwerte schließen lässt.