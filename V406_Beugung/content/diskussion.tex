\section{Diskussion}
\label{sec:Diskussion}
\subsection{Fehlerrechnung}
    Zur Bestimmung der relativen Fehler wurde die Formeln
    \begin{equation*}
        \dfrac{\Delta x}{x}=\dfrac{x_{theo}-x_{ex}}{x_{theo}}
    \end{equation*}
    verwendet, wobei $x_{theo}$ die Herstellerangaben darstellen und $x_{ex}$ die 
    experimentell bestimmten Werte.

\subsection{Einzelspalt}
    Der Hersteller gibt für den Einzelspalt eine Breite von 0.15 mm an. In der Auswertung wird
    die Spaltbreite zu 0.157 mm bestimmt. Somit liegt eine Abweichung um $ 4.7 \% $ vor. Er zeigt, dass die 
    Herstellerangaben sowie die verwendeten Formeln weitestgehend funktionieren.
    
\subsection{Doppelspalt}
    Die Herstellerangaben zum Doppelspalt lauten:
    \begin{align*}
        Breite = 0.15 \si{\milli\metre}\\
        s = 0.65 \si{\milli\metre}\\
    \end{align*}
    Diese weichen von den tatsächlich bestimmten Werten 
    \begin{align*}
        Breite = 0.138 \pm 0.45 \si{\milli\metre}\\
        s = 0.498 \pm 0.1 \si{\milli\metre}
    \end{align*}
    ab. Die Abweichung der Breite des Spaltes wird zu 8.19\% bestimmt. Jedoch liegen
    die Herstellerangaben zur Breite innerhalb der bestimmten Unsicherheit. Die
    Herstellerangaben für s liegen jedoch gerade außerhalb der Unsicherheit. Folgende 
    Fehlerquellen werden ausgemacht:\\
    Es könnte sein, dass 
    der Spalt während der Messungen verutscht ist und somit die Messergebnisse verfälscht werden.
    Außerdem konnte das Amperemeter nicht auf die niedrigste Messskala eingestellt werden, weshalb
    exakte Messungen erschwert wurden. Des weiteren kommen als Fehlerfaktoren auch noch der 
    Dunkelstrom und Restlicht, verursacht durch eine offenstehende Tür und eine Steckdosenleiste, 
    ins Spiel.
    Außerdem gibt es erhebliche Defizite beim Curvefit. Der Spalt des Detektors ist einen Millimeter 
    breit, weshalb die Extrema des Doppelspaltes nicht gut aufgelöst werden können. Dies kann beim 
    fitten der Funktion zu enormen Fehlern führen.

