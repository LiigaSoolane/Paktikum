\section{Theorie}
\label{sec:Theorie}

Ziel des Versuches ist es, das Elastizitätsmodul verschiedener Materialien zu bestimmen.

\subsection{Das Elastizitätsmodul}
    Die physikalische Größe Spannung beschreibt die Kraft F, die an der Fläche A
    angreift. Unterschieden wird zwischen der Normalspannung $\sigma$, welche 
    senkrecht auf der Oberfläche steht, und der Tangential- oder Schubspannung, die parallel 
    zur Oberfläche wirkt. Durch Spannung können Gestalts- und Volumenveränderung entstehen.
    Liegt die Verformung nur in einer Körperdimension vor und ist die relative 
    Verformung $\dfrac{\Delta L}{L}$ hinreichend klein, tritt das Hooksche Gesetz
    \begin{equation}
        \label{eqn:Hook}
        \sigma = E \dfrac{\Delta L}{L}
    \end{equation}
    mit dem Elestizitätsmodul $E$ in Kraft. Um dieses für verschiedene Materialien zu bestimmen, 
    wird die Biegung von Stäben betrachtet, wie in Abb. ... und Abb. zu sehen.

\subsection{Mechanische Spannung bei Biegung}
    \label{sec:aiaiai}
    Die Biegung eines Stabes verursacht Spannung innerhalb der Probe. Wie in Abbildung ...
    dargestellt kommt es zu Dehnung der äußeren Fasern und Stauchung der inneren. 
    Allein die neutrale Faser in der Mitte des Querschnitts Q behält ihre Länge bei.
    Bei den anderen Fasern kommt es aufgrund der Elastizitäts des Körpers zu Normalspannungen.
    Aus geometrischen Günden, die in Abbildung ... ersichtlich sind, ergibt sich für die 
    Änderung $delta x$ einer Faser der Länge $\Delta x$
    \begin{equation*}
        \delta x = y \Delta \phi= y\dfrac{\Delta x}{R},
    \end{equation*}
    wobei y den Abstand der betrachteten zur neutralen Faser und R den Biegeradius
    symbolisiert. Anhand des Hookschen Gesetzes kann nun die durch die Biegung 
    entstehende Normalspannung bestimmt werden.
    \begin{equation*}
        \sigma(y)=E\dfrac{\delta x}{\Delta x}=E \dfrac{y}{R}
    \end{equation*}
    Auf einen Querschnitt Q wirken diese Spannungen wie ein Drehmoment $M_{\sigma}$ mit
    Hebelarm y. Das Geseamtdrehmoment, welches auf Querschnitt Q wirkt, lässt sich also 
    zu
    \begin{equation}
        \label{eqn:fuckyou}
        M_{\sigma}=\int_Q y \sigma(y)dq = \int_Q y^2 \dfrac{E}{R}dq 
    \end{equation}
    aufsummieren. 
    
\subsection{Biegung bei einseitiger Einspannung}
    Wird ein Stab einseitig eingespannt und ein Gewicht am uneingespannten Ende des Stabes 
    installiert, biegt sich der Stab aufgrund der Gravitation mit der Durchbiegung D(x) 
    und bewirkt am Ort x ein angreifendes Drehmoment
    \begin{equation}
        M_F = F \cdot (L − x)
    \end{equation}
    mit der Länge des Hebelarms (L − x).
    Dieses Drehmoment verursacht eine Biegung des Stabes, wodurch die im Kapitel 
    \ref{sec:aiaiai} berechneten Normalspannungen zustande kommen. Diese heben das 
    Drehmoment exakt auf, so dass ein Kräftegleichgewicht zwischen $M_F$ und $M_{\sigma}$ herrscht.
    \begin{equation}
        \int_Q y^2 \dfrac{E}{R}dq  = F \cdot (L − x)
    \end{equation}
    Der Krümmungsradius R kann durch $\dfrac{1}{R} \approx 
    \dfrac{d²D}{dx²}$ approximiert werden, sodass sich für Gleichung \ref{eqn:fuckyou}
    \begin{equation}
        E \dfrac{d²D}{dx²} \int_Q y² dq=F\cdot(L-x)
    \end{equation}
    ergibt. Die Durchbiegung kann also zu
    \begin{equation}
        D(x)=\dfrac{F}{2 E \ I}(Lx²-\dfrac{x^3}{3})
    \end{equation}
    mit Flächenträgheitsmoment
    \begin{equation}
        I = \int_Q y² dq
    \end{equation}
    bestimmt werden.

\subsection{Biegung bei zweiseitiger Auflage}
    Wie in Abb. ... dargestellt, wird bei beidseitiger Auflage ein Gewicht in der Mitte des 
    jeweiligen Stabes installiert, sodass dieser sich zur Mitte hin biegt. Bei dieser 
    Konstruktion greift am Ort x die Kraft $\dfrac{F}{2}$ mit Hebelarm x 
    beziehungsweise (L-x) auf der linken Seite an der Querschnittsfläche Q
    an. Auf die Querschnittsfläche wirkt also das Drehmoment $M_F = -\dfrac{F}{2} x$ 
    beziehungsweise $M_F = -\dfrac{F}{2}(L-x)$ auf der linken Seite des Stabes.
    Somit erhalten wir 
    \begin{equation}
        \text{für } 0 \leq x \leq \dfrac{L}{2}:\ \ \ \ \dfrac{d D}{dx}=-\dfrac{F}{E\ I}\dfrac{x^2}{4}+C \\
        \text{für} \dfrac{L}{2} \leq x \leq L:\ \ \ \ \dfrac{d D}{dx}=-\dfrac{F}{2 E\ I}(Lx - \dfrac{x^2}{2})+C'
    \end{equation}
    Da die Tangente in der Mitte des Stabes horizontal ist, gilt für 
    \begin{equation*}
        C = \dfrac{F}{E\ I}\dfrac{L^2}{16} \\
        C'=\dfrac{F}{E\ I}\dfrac{3L^2}{16}
    \end{equation*}
    Somit folgt für 
    \begin{align}
        \text{für } 0 \leq x \leq \dfrac{L}{2}\ \ \ \ D(x)=\dfrac{F}{48 E\ I}(3L^2x-4x^3) \\
        \text{für} \dfrac{L}{2} \leq x \leq L:\ \ \ \ D(x)=\dfrac{F}{48 E\ I}(4 x^3-12Lx^2+9L^2x-L^3) 
    \end{align}

%\subsection{Zusammenhang zwischen Elastizitätsmodul und Schallgeschwwindigkeit in Festkörpern}
%    Läuft eine Schallwelle durch einen Körper, so entsteht durch die longitudinale Deformation
%    Spannung. Ein Volumenelement $dV=Q\ dx$ an Ort x wird durch die Spannung $\sigma(x)$
%    um $\eta$ verschoben und um $d \eta$ degehnt, da an Ort x eine andere Spannung $\sigam(x)$
%    herrscht als bei x+dx, wo $\sigma(x)+d\sigma$ auftritt. Nach dem Hookschen Gesetz gilt:
%    \begin{equation*}
%        \sigma=E\dfrac{\partial \eta}{\partial x}
%    \end{equation*}
%    beziehungsweise
%    \begin{equation*}
%        \sigma + d \sigma = E \dfrac{\partial \eta}{\partial x}+E\dfrac{\partial^2 \eta}{\partial x^2}dx
%    \end{equation*}
%    Somit wirkt auf ein Volumenelement Q dx die Kraft
%    \begin{equation}
%        dF=Q((\sigma + d\sigma)-\sigma)=Q \ E\dfrac{\partial^2 \eta}{\partial x^2}dx
%    \end{equation}
%    Daraus lässt sich schlussfolgern, dass
%    \begin{equation}
%        \dfrac{\partial^2 \eta}{\partial t^2}=\dfrac{E}{\rho} \dfrac{\partial^2 \eta}{\partial x^2}
%    \end{equation}
%    Diese Wellengleichung schreibt der Ausbreitungsgeschwindigkeit $ c = \sqrt{\dfrac{E}{\rho}}$ zu.
%    Dies kann ebenfalls verwendet werden, um das Elastizitätsmodul E zu bestimmen.