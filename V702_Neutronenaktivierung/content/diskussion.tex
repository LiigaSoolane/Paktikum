\section{Diskussion}
\label{sec:Diskussion}

\subsection{Vanadium}

Die Halbwertzeit von Vanadium wurde zunächst mit allen aufgenommenen Werten berechnet, dabei ergab sich ein Wert
von $T_1 = 215 \pm 7 \si{\s}$, die genauere Methode, die nach der doppelten Halbwertszeit abbricht, ergab 
$T_2 = 208 \pm 13 \si{\s}$. Nach Quelle %\cite{}
ist der Theoriewert $T_s = 224 \si{\s}$, was zu einer prozentualen Abweichung für $T_1$ von etwa 9 Prozent und für 
$T_2$ etwa 16 Prozent führt. Ungewöhnlich ist dabei, dass die Abweichung, die bei der prinzipiell genaueren Methode herauskommt,
größer ist als bei der prinzipiell ungenaueren Methode. 
Der größere Fehler des Wertes lässt sich dadurch erklären, dass durch eine geringere Anzahl an Werten auch ein größerer statistischer Fehler entsteht. 
Außerdem ist die generelle Unsicherheit des Versuches recht hoch, beispielsweise durch die Auswirkungen einer möglicherweise nicht idealen Wahl von $\increment t$
für die Messung. Durch ein zu kleines Zeitintervall kann ein statistischer Fehler dazu kommen, während ein zu großes Zeitintervall zu einem systematischen Fehler in 
$\lambda$ führen kann. \\
Zudem ist mit dem Zählrohr, das für die Messung des Zerfalls genutzt wurde, keine einwandfreie Messung möglich, da nach jeder Detektion für eine Totzeit keine Werte 
aufgenommen werden können. Dieser Effekt kommt zu Anfang de Messung mehr zum Tragen, da dann mehr Impulse in einem Zeitintervall aufgenommen werden.\\
Eine Verbesserung der Werte wäre außerdem durch eine längere Messung der Untergrundstrahlung möglich, um den statistischen Fehler zu verringern, oder durch bessere Abschirmung
des Geiger-Müller-Zählrohres.

\subsection{Rhodium}

In der zweiten Hälfte des Versuches wurde für den Zerfall des langlebigeren Isotopes %welches isses denn jetzt?
eine Halbwertszeit von $258.14 \pm 0.001 \si{\s}$ berechnet, während sich für das kurzlebige ein Wert von $43.054 \pm 0.001 \si{\s}$ ergab. In Quelle %\cite{}
sind die Theoriewerte für die beiden Isotope zu finden, sie lauten $T_{\text{theo, l}} = 260 \si{\s}$ für das langlebige und $T_{\text{theo, k}} = 41 \si{\s}$ für
das kurzlebige. \\
So ergeben sich prozentuale Abweichungen von $1.8$ Prozent für %ajsglaj
und 2 Prozent für das Andere, also in beiden Fällen sehr geringe Abweichungen. In Abbildung \ref{fig:regress}, die die entsprechenden Kurven darstellt, ist die Güte dieser 
Werte auch erkennbar, da die zusammengesetzte Kurve genau durch die Datenpunkte verläuft. \\
Geringe Abweichungen bei den einzelnen Kuven für die jeweiligen Isotope kommen dadurch zustande, dass der Einfluss des andern Isotops bei der realen Messung natürlich auch noch
einen Einfluss hat, sodass die Werte nicht perfekt repräsentiert sind. Dies ist erst bei Kombination der Kurven der Fall. 