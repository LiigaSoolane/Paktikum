\section{Diskussion}
\label{sec:Diskussion}
Die prozentualen Abweichungen werden mithilfe der Formel
\begin{equation*}
    \increment x = \frac{x_{\symup{theoretisch}} - x_{\symup{gemessen}}}{x_{\symup{theoretisch}}}
\end{equation*}
berechnet.

\subsection{Bragg-Bedingung}
    Der gemessene Glanzwinkel liegt bei $\theta_{\text{max}} = 28.2 °$, die Abweichung vom Theoriewert von $\theta_{\text{theo}} = 28 °$ beträgt $0.713 \%$.
    Wäre die Abweichung größer als drei Grad, würde die Messung unbrauchbar werden: anstatt der gewollten Zählrate würde man einen 
    Untergrund aufnehmen.

\subsection{Emissionssprektrum}
    \subsubsection{K-Linien}
        Die in Kapitel \ref{sec:Auswertung} berechneten Werte lauten
        \begin{align*}
            E_{K_{\alpha}} = 8044 \pm 34 \text{eV}\\
            E_{K_{\beta}} = 8910 \pm 40 \text{eV}
        \end{align*}
        Die Theoriewerte betragen (Quelle \cite{nist})
        \begin{align*}
            E_{K_{\alpha}\text{, theo}} = 8048.11 \pm 45 \text{eV}\\
            E_{K_{\beta}\text{, theo}} = 8906.9 \pm 12 \text{eV}
        \end{align*}
        woraus eine prozentuale Abweichung von $0.1 \pm 0.7 \%$ für die $K_{\alpha}$ Linie und $0.1 \pm 0.5 \%$ für die $K_{\beta}$ Linie folgt. Beide Abweichungen sind also 
        ausgesprochen gering und deuten auf genaue Messungen hin.\\
        Dem Messverfahren unterliegt eine gewisse statistische Messunsicherheit, die bei jeder Messung eine Rolle spielt. Da es sich um zur Verfügung gestellte Messwerte handelt,
        kann die Situation während der Datenaufnahme nicht in Betracht gezogen werden.

    \subsubsection{Auflösungsvermögen}
        Aus der Halbwertsbereiten folgen Auflösungsvermögen von 
        \begin{align*}
            A_{K_{\alpha}} = (7.638 \pm 0.032) \cdot 10^{-18} \\
            A_{K_{\beta}} = (6.788 \pm 0.032) \cdot 10^{-18}
        \end{align*}
        

\subsection{Absorptionsspektren}
    \subsubsection{Spektren der verschiedenen Elemente}
    In Tabelle \ref{tab:energies} sind die berechneten Absorptionsenergien sowie die Theoriewerte und prozentualen Abweichungen aufgeführt. 
    \begin{table}
        \centering
        \caption{Mess- und Theoriewerte zur Absorptionsenergie.}
        \label{tab:energies}
        \begin{tabular}{c c c c c}
            \toprule 
            Element & $Z$ & berechneter Wert/eV & Theoriewert/eV & Abweichung/\% \\
            \midrule
            Zink & 30 & $9600 \pm 25$ & 9668.55 & $0.7 \pm 2.6$ \\
            Gallium & 31 & $10320 \pm 29$ & 10377.76 & $0.5 \pm 2.8$ \\
            Brom & 35 & $13500 \pm 50$ & 13483.86 & $0 \pm 4$ \\
            Rubidium & 37 & $15100 \pm 60$ & 15207.74 & $1 \pm 4$ \\
            Strontium & 38 & $16000 \pm 70$ & 16115.26 & $1 \pm 4$ \\
            Zirkonium & 40 & $17700 \pm 90$ & 18008.15 & $2 \pm 5$ \\
            \bottomrule
        \end{tabular}
    \end{table}
    \FloatBarrier

    \noindent Bis auf für die Rechnung von Zirkonium liegen alle Abweichungen unter $10\%$, wenn auch bei manchen nur knapp. Wie oben bereits erwähnt ist der statistische 
    Fehler eine mögliche Ursache, besonders da nur wenige Werte aufgenommen wurden. Bei einer größeren Anzahl an Messwerten würde dieser Fehler deutlich kleiner werden.\\
    In Tabelle \ref{tab:abschirm} sind die Theorie- und Messwerte der Abschirmkonstanten für alle Elemente aufgeführt. Die Theoriewerte wurden dabei aus den Energien 
    in Tabelle \ref{tab:energies} berechnet.

    \begin{table}
        \centering
        \caption{Mess- und Theoriewerte zu den Abschirmkonstanten.}
        \label{tab:abschirm}
        \begin{tabular}{c c c c c}
            \toprule 
            Element & $Z$ & berechneter Wert & Theoriewert & Abweichung [\%] \\
            \midrule
            Zink & 30 & $3.63 \pm 0.35$ & 3.54 & $3 \pm 10$ \\
            Gallium & 31 & $3.7 \pm 0.4$ & 3.6 & $2 \pm 11$ \\
            Brom & 35 & $3.8 \pm 0.6$ & 3.831 & $0 \pm 15$ \\
            Rubidium & 37 & $4.1 \pm 0.7$ & 3.935 & $4 \pm 18$ \\
            Strontium & 38 & $4.1 \pm 0.8$ & 3.983 & $3 \pm 19$ \\
            Zirkonium & 40 & $4.4 \pm 0.9$ & 4.083 & $7 \pm 22$ \\
            \bottomrule
        \end{tabular}
    \end{table}
    \FloatBarrier

    \noindent Genrell liegen die Abweichungen unter $10 \%$, allerdings sind sie von großer Ungenauigkeit betroffen.\\
    Was besonders auffällt, ist, dass die Abweichungen und deren Unsicherheiten für steigende $Z$ größer werden. Dies 
    kommt daher, dass in der zuvor durchgeführten Rechnung der Einfluss der Feinstruktur weitgehend vernachlässigt, bzw.
    vereinfacht wurde. Für Atome größerer Ordnungszahlen wird dieser Effekt allerdings immer bedeutender,
    sodass die Rechnung keine genauen Werte mehr liefern kann.\\

    \subsubsection{Rydbergenergie}
        Für die Rydbergenergie und -frequenz folgen aus den Messwerten
        \begin{align*}
            R = (3 \pm 0.05) \cdot 10^{15} \si{\Hz}\\
            R_{\infty} = 12.42 \pm 0.19
        \end{align*}
        was einer prozentualen Abweichung vom Theoriewert $R_{\infty \text{, theo}} = 13.6 \text{eV}$ von $8.7 \pm 1.4 \%$ entspricht. \\
        Die Ungenauigkeiten aus den vorherigen Messungen kommen zusammen und verstärken sich, sodass hier eine größere Unsicherheit auftritt, die auch in größeren Unterschieden zu
        Theoriewerten zum Ausdruck kommen kann.\\
        Die Messunsicherheiten des Verfahrens wurden bereits diskutiert, hier kommt noch die Ungenauigkeit aus der linearen Regression hinzu.