\section{Durchführung}
\label{sec:Durchführung}

\subsection{a) }
    Zunächst wird die in \ref{sec:Theorie} erklärte Wheatonsche Brückenschaltung aufgebaut, um mit ihr einen unbekannten Widerstand zu ermitteln. 
    Abgeglichen wird dabei mit einem Potentiometer mit 1 kOhm Gesamtwiderstand. Der feste ohmsche Widerstand $R_2$ wird dabei variiert, um die 
    Messung genauer zu machen und die Fehler berechnen zu können.\\
    Zur Messung wird ein Millivoltmeter genutzt.

\subsection{b) }
    Des weiteren wird eine Kapazitätsmessbrücke gemäß \ref{fig:kap} aufgebaut und verwendet, um die Kapazität und den Widerstand eines realen
    Kondensators zu ermitteln. Dazu werden die regelbaren Widerstände abwechselnd so eingestellt, das die Brückenspannung ein Minimum erreicht.

\subsection{c) }
    Um die Induktivität einer realen Spule zu berechnen, wird eine Induktivitätsmessbrücke aufgebaut. Das Messprinzip ist genau wie bei der Kapazitätsmessbrücke,
    auch hier werden die Widerstände alternierend verstellt.

\subsection{d) }
    Dieselbe Induktivität soll ein weiteres Mal gemessen werden, diesmal unter Verwendung der Maxwell-Brücke (\ref{fig:maxwell}).\\
    Dazu wird die Schaltung nach dem Schaltbild aufgebaut und alternierend mit den beiden Widerständen abgeglichen.

\subsection{e) }
    Zuletzt soll die Wien-Robinson-Brücke auf ihre Frequenzabhängigeit untersucht werdem, dazu wird die Brückenspannung gemessen für Frequenzen im Bereich 
    von $20 \si{\k\Hz} - 20 000 \si{\k\Hz}$. Beim Nähern der Werte an die minimale Frequenz wird der Abstand zwischen den Messpunkten verkleinert, sodass 
    ein annähernd logarithmischer Datensatz entsteht.