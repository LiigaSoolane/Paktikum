\section{Diskussion}
\label{sec:Diskussion}

\subsection{Einzelspalt}
    Der Hersteller gibt für den Einzelspalt eine Breite von 15 mm an. In der Auswertung wird
    die Spaltbreite zu 15.7 mm bestimmt. Somit liegt eine Abweichung um $\dfrac{15.7 \si{\milli\metre} 
    - 15 \si{\milli\metre}}{15 \si{\milli\metre}} = 4.7 $ vor. Für einen Versuch des Physikpraktikums mit den veralteten Geräten 
    und in diesem Fall instabilen Versuchsaufbau ist dieser Wert sehr gut. Er zeigt, dass die 
    Herstellerangaben sowie die verwendeten Formeln weitestgehend funktionieren.
    
\subsection{Doppelspalt}
    Die Herstellerangaben zum Doppelspalt lauten:
    \begin{align*}
        Breite = 0.15 \si{\milli\metre}\\
        s = 0.65 \si{\milli\metre}\\
    \end{align*}
    Diese weichen von den tatsächlich bestimmten Werten 
    \begin{align*}
        Breite = (1.38 \pm 0.45)\cdot 10^{-4} \si{\metre}\\
        s = (4.98 \pm 0.1) \cdot 10^{-5} \si{\metre}
    \end{align*}
    ein ab. Die Abweichung der Breite des Spaltes wurde zu 8.19\% bestimmt. Jedoch liegen
    die Herstellerangaben zur Breite innerhalb der bestimmten Unsicherheit. Die
    Herstellerangaben für s liegen jedoch gerade außerhalb der Unsicherheit. Folgende 
    Fehlerquellen wurden ausgemacht:\\
    Es könnte sein, dass 
    der Spalt während der Messungen verutscht ist und somit die Messergebnisse verfälscht werden.
    Außerdem konnte das Amperemeter nicht auf die niedrigste Messskala eingestellt werden, weshalb
    exakte Messungen erschwert wurden. Des weiteren kommen als Fehlerfaktoren auch noch der 
    Dunkelstrom und Restlicht, verursacht durch eine offenstehende Tür und eine Steckdosenleiste, 
    ins Spiel.
    Außerdem gibt es erhebliche Defizite beim Curvefit. Der Spalt des Detektors ist einen Millimeter 
    breit, weshalb die Extrema des Doppelspaltes nicht gut aufgelöst werden können. Dies kann beim 
    fitten der Funktion zu enormen Fehlern führen.

