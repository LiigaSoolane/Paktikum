\section{Theorie}
\label{sec:Theorie}

\subsection{Zielsetzung}
    Das Ziel dieses Versuches ist es, die quantenmechanische Beschreibung der Elektronenhülle
    näher zu umreißen. Dazu werden Stöße von Elektronen und Atomen herbeigeführt und
    der Energieverlust der Elektronen betrachtet. Im Franck-Hertz-Versuch wechselwirken
    monoenergetische Elektronen mit Hg-Dampf. Dabei finden sowohl elastische als auch inelastische
    Stöße statt. Bei inelastischen Stößen wird das Hg-Atom aus dem Grundzustand der Energie $E_0$ 
    in den angeregten Zustand $E_1$ angehoben. Aus der Energieerhaltung folgt:
    \begin{equation}
        \dfrac{m_0\cdot v_{vor}^2}{2}-\dfrac{m_0\cdot v_{nach}^2}{2}=E_1-E_0
    \end{equation}
    wobei $m_0$, $v_{vor}$ und $v_{nach}$ die Masse und die Geschwindigkeiten des Elektrons
    bezeichnen. Konkret wird die Energiedifferenz zwischem
    Grund- und nächsthöheren angeregten Zustand von Hg-Atomen bestimmt sowie die Energieverteilung
    der hier verwendeten Elektronen näher betrachtet. 

\subsection{Aufbau}
    Der Versuchsaufbau ist in der folgenden Abbildung ... dargestellt. In einer evakuierten Glasröhre
    befinden sich Heizfaden, Beschleunigungs- und Auffängerelektrode und ein Tropfen 
    Quecksilber, der beiErhitzung der Apperatur gemäß der Dampfdruckkurze spontan 
    verdampft. Wird die Temperatur konstant gehalten, stellt sich ein Gleichgewichtsdampfdruck 
    $p_{sät}$ ein, der sich für Quecksilber durch
    \begin{equation}
        p_{sät}(T)=5.5\cdot 10^{7} \ \exp{-\dfrac{6876}{T}}
    \end{equation}
    berechnen lässt ($p_{sät}$ wird in mbar angegeben).\\
    \noindent Der Heizfaden besteht aus einem hochschmelzendem Metall, welches durch Gleichstrom
    bis zur Rotglut erhitzt wird. Dadurch tritt eine große Zahl an Elektronen aus, die durch die 
    positiv geladene Elektrode in der Mitte des Glasröhre beschleunigt werden. Die kinetische Energie
    nach Passieren der Beschleunigungsstrecke beträgt
    \begin{equation}
        \dfrac{m_0\cdot v_{vor}^2}{2}=e_0 U_B.
    \end{equation}
    Hierbei stellt U_B die Beschleunigungsspannung dar, die zwischen Heifaden und 
    Beschleunigungselektrode angebracht ist.\\
    \noindent Am Ende der Glaröhre befindet sich eine Auffängerelektrode. Der durch die aufgefangenen
    Elektronen entstandene Strom $I_A$ wird mittels eines ... gemessen. Da die Auffängerelektrode
    negativ geladen ist gelangen nur solche Elektronen in ihre Nähe, die die Ungleichung 
    \begin{equation}
        \dfrac{m_0}{2}v_Z^2 \geq e_0 U_A
    \end{equation}
    erfüllen mit der Geschwindigkeit v_Z des Elektrons in Feldrichtung, und der Elektrodenspannung
    $U_A$.\\
    Auf dem Weg von Heizfaden zu Auffängerelektrode finden Stöße mit denen sich in der 
    Glasröhre befindlichen Hg-Atomen. Bei niedrige Elektronenenergien finden nur 
    elastische Stöße statt. Im zentralen Stoß beträgt die Energieabgabe des Elektrons
    \begin{equation}
        \Delta E = \dfrac{4 m_0 M}{(m_0 + M)^2}E \approx 1.1\cdot 10^{-5}E,
    \end{equation}
    ist also sehr gering, was an dem großen Massenunterschied zwischen Elektron und 
    Hg-Atom liegt. Die Geschwindigkeitsveränderung kann also vernachlässigt werden,
    die Richtungsänderung ist jedoch für den Franck-Hertz Versuch von Relevanz.\\
    Bei Elektronenenergien, die über der Energiedifferenz $E_1-E_0$ liegen, ist es 
    dem Elektron möglich, das gestoßene Hg-Atom anzuregen und dabei den entsprechenden 
    Betrag an Energie zu verlieren. Das angeregte Hg-Atom fällt mit einer Relaxationszeit
    von $10^{-8}$ s wieder in den 
    Grundzustand zurück und emittiert dabei ein Lichtquant der Energie
    \begin{equation}
        h\nu = E_1-E_0 .
    \end{equation}
    Wird $U_A$ festgehalten, während $U_B$ variiert wird, entwickelt sich der Auffängerstrom
    wie in der idealisierten Abbilgung ... aufgetragen. Überschreitet $U_B$ $U_A$, so 
    nimmt der Elektronenstrom stark zu, bis die Elektronen durch die Beschleunigung die
    kritische Energie von $E_1-E_2$ überschreiten, und diese dann in unelastischen Stößen
    an die Hg-Atome abgeben. Der Spannungsunterschied zwischen den Maxima beträgt somit
    \begin{equation}
        U_1 = \dfrac{1}{e_0}(E_1-E_0).
    \end{equation}

\subsection{Weitere Einflüsse}
    \subsubsection{Kontaktpotential}
        Durch Effekte wie den elektrischen Widerstand gibt es eine erhebliche Differenz zwischen 
        angelegtem und tatsächlichem Beschleunigungspotentiol. Besonders deutlich treten diese
        Effekte zu Tage, wenn beide Elektroden unterschiedliche Austrittsarbeit für Elektronen 
        besitzen, was bei diesem Versuch offensichtlich sehr hilfreich ist. Das effektive 
        Beschleunigungspotential kann durch 
        \begin{equation}
            U_{b, eff} = U_B - K
        \end{equation}
        mit dem Kontaktpotential
        \begin{equation}
            K = \dfrac{1}{e_0}(\Phi_B - \Phi_G)
        \end{equation}
        bestimmt werden.
    \subsubsection{Energie-Spektrum der Elektronen}
        Schon bei der Glühemission besitzen die Elektronen unterschiedliche Geschwindigkeiten.
        Nach Durchlauf der Beschleunigungsstrecke besitzen diese also ein kontinuierliches
        Energiespektrum. Daher setzen die unelastischen stöße nicht bei einer bestimmten 
        Beschleunigungsspannung ein und fallen auch nicht schlagartig bei einer anderen 
        Spannung wieder ab. \\
        Zudem sind auch die elastischen Stöße in diesem Zusammenhang nicht vernachlässigbar. 
        Sie ändern die Energie der Elektronen zwar nicht übermäßig, aber die Richtung, wodurch
        sich die Frack-Hertz-Kurve weiter abflacht.
    \subsubsection{Dampfdruck}
        Um Resultate aus dem Versuch gewinnen zu können, müssen genügend Zusammenstöße von 
        Elektronen und Hg-Atomen stattfinden. Aus diesem Grund muss die mittlere freie 
        Weglänge $\overline{w}$ klein sein (dh. Faktor 1000 bis 4000 für ausreichende
        Stoßwahrscheinlichkeit) gegen den Abstand von Kathode und Beschleunigungselektrode.
        Durch die kinetische Gastheorie kann für die mittlere freie Weglänge der Zusammenhang
        \begin{equation}
            \overlin{w} [cm]=\dfrac{0.0029}{p_{sät}}[p in mbar]
        \end{equation}
        hergeleitet werden. Wählt man $p_{sät}$ alleerdings zu hoch, finden zu viele elastische
        Stöße statt, was der Messung schadet.
    \subsubsection{Aufbau der Hg-Atomhülle}
        Für diesen Versuch relevant sind nur die s s-Elektronen, die sich in der Schale mit 
        Hauptquantenzahl $n=6$ befinden. AUfgrund des Pauli-Prinzipes lauten die 
        Quantenzahlen der Elektronen im Grundzustand:
        \begin{align*}
            & n_1=n_2=6\ & l_1=l_2=0 \ & s_1 = -s_2=\dfrac{1}{2}
        \end{align*}
        Somit verchwindet die Gesamtspinzahl S und der Grundzustand besitzt keine Feinstruktur.
        Die für ein Elektron eines solchen Atoms möglichen angeregten Zustände sind in der folgenden
        Grafik dargestellt

        Der dort abgebildete Übergang vom Grundzustand in den ersten angeregten Zustand ist 
        selten - so lange man das Atom nicht mit Elektronen beschießt, wie in diesem Versuch.


