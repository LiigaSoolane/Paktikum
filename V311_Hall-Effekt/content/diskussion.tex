\section{Diskussion}
\label{sec:Diskussion}
Die Messwerte, die im Laufe des Versuches aufgenommen und ausgerechnet wurden,
lauten
\begin{align*}
R = 2.65 \si{\ohm} \\
\rho = 1.56 \cdot 10^{-8} \si{\ohm\m} \\
n_\text{Sk} = \left( 1.271 \pm 0.022 \right) \cdot 10^{29} \si{\per\m\cubed}\\
n_\text{Pk} = \left( 0.899 \pm 0.19 \right) \cdot 10^{29} \si{\per\m\cubed}\\
z_\text{Sk} = 1.501 \pm 0.026 \\
z_\text{Pk} = 1.061 \pm 0.023 \\
v_\text{d, Sk} = (4.91 \pm 0.09) \cdot 10^{-5} \si{\m\per\s}\\
v_\text{d, Pk} = (6.95 \pm 0.15)  \cdot 10^{-5} \si{\m\per\s}\\
\bar{\tau}_{Sk} = \left( 3.59 \pm 0.06 \right) \cdot 10^{-14} \si{\s}\\
\bar{\tau}_{Pk} = \left( 5.08 \pm 0.11 \right) \cdot 10^{-14} \si{\s}\\
\mu_{Sk} = \left( 3.16 \pm 0.06 \right) \cdot 10^{-2} \si{\m\squared\per\volt\s} \\
\mu_{Pk} = \left( 4.47 \pm 0.09 \right) \cdot 10^{-2}\si{\m\squared\per\volt\s} \\
\bar{v}_{Sk} = \left( 1.801 \pm 0.011 \right) \cdot 10^{6} \si{\m\per\s}\\
\bar{v}_{Pk} = \left( 1.604 \pm 0.011 \right) \cdot 10^{6} \si{\m\per\s}\\
\bar{l}_{Sk} = \left( 6.47 \pm 0.08 \right) \cdot 10^{-8} \si{\m}\\
\bar{l}_{Pk} = \left( 8.15 \pm 0.12 \right) \cdot 10^{-8} \si{\m}
\end{align*}

Tabelle \ref{tab:theo} zeigt die Theoriewerte und die prozentualen Abweichungen 
der jeweiligen Messreihen. Dabei stammen die Werte für die Ladungsträgersichte und
die Beweglichkeit aus Quelle \cite{beweglichkeit}, die für die Totalgeschwindigkeit,
die mittlere Flugzeit und die mittlere freie Weglänge aus \cite{transport}. 
\begin{table}[H]
 \centering
 \caption{Theoriewerte und prozentuale Abweichungen.}
 \label{tab:theo}
 \begin{tabular}{lccc}
  \toprule
   & & \multicolumn{2}{c}{Abweichungen in Prozent}\\
  \cmidrule(lr){3-4}
  Messwert & Theoriewert & Spulenstrom konst. & Probenstrom konst. \\
  \midrule
  Ladungsträgersichte n & $1 \cdot 10^{29} \si{\per\m\cubed}$ & 2.7 & 1.0\\
  $\bar{\tau}$ & $2.51 \cdot 10^{-14} \si{s}$ & 4.3 & 10.2\\
  Beweglichkeit $\mu$ & $4.36 \cdot 10^{-3} \si{\m\squared\per\volt\s}$ & 2.7 & 0.2 \\
  Totalgeschwindigkeit $\bar{v}$ & $1.6 \cdot 10^{6} \si{\m\per\s}$ & 1.2 & 0.3 \\
  mittlere freie Weglänge & $4.02 \cdot 10^{-8} \si{\m}$ & 6.1 & 10.2 \\
  \bottomrule
 \end{tabular}
\end{table}
\noindent Der Theoriewert für den spezifischen Widerstand beträgt nach \cite{spezid} 
$\rho_\text{Theorie} = 1.75 \cdot 10^{-8} \si{\ohm\m}$, also folgt eine prozentuale 
Abweichung von 10 Prozent.\\
Die Abweichungen sind im Allgemeinen gering, dennoch lässt sich eine Diskrepanz feststellen.
Eine mögliche Ursache für Messunsicherheiten ist, dass die Messung des Stroms sehr ungenau
war, da die Ströme sehr klein waren und schon im Millivolt-Bereich nur 
die dritte Nachkommastelle gemessen wurde. Diese Ungenauigekeit zieht sich
natürlich durch die komplette Rechnung durch.\\
Zudem hat es der Generator nicht immer geschafft, auf $5 \si{\A}$ hochzuregeln
und so sind die Werte, mit denen für 5 Ampere gerechnet wurde, nicht immer einwandfrei
vergleichbar. \\
Da digitale Messgeräte zur Messung der Spannung und des Magnetfeldes verwendet
wurden, ist zu erwarten, dass der menschliche Fehler bei diesen Werten eine geringere
Rolle spielt.\\
Da bei der Herleitung der Formeln einige Näherungen gemacht wurden, sind ebenfalls
Abweichungen zu Erwarten.
Zwischen den jeweiligen Messreihen ist kein deutlicher Unterschied festzustellen.