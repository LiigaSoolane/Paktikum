\section{Diskussion}
\label{sec:Diskussion}

Die Abweichung der aus den Widerständen und Spannungen bestimmten der Brückenschaltung bestimmten 
Suszibilitäten von den Theoriewerten ist in der folgenden Tabell dargestellt.
\begin{table}[H]
    \centering
    \caption{Suszeptibilitäten}
    \label{tab:mag}
    \begin{tabular}{c c c}
     \toprule
      & $\dfrac{\chi_{theo}-\chi_U}{\chi_{theo}}$ & $\dfrac{\chi_{theo}-\chi_R}{\chi_{theo}}$\\
     \midrule
      Gd$_2$O$_3$ & 0.771 & 0.911 \pm 0.0035 \\
       Dy$_2$O$_3$ & 0.872 & 0.951 \pm 0.002 \\
     \bottomrule
    \end{tabular}
   \end{table} 

Die Theoriewerte unterscheiden sich um eine Größenordnung von den durch die Messungen gewonnenen 
Werten. \\ 
Dies kann unter Anderem darauf zurückgeführt werden, dass die Abmessungen der Probe nicht
mit den Gegebenheiten innerhalb der Spule übereinstimmen, da die Probe nicht vollständig in 
die Messapperatur eingeführt werden kann.\\
Außerdem wurden erhebliche Mängel an dem Selektivverstärker oder am Sinusgenerator festgestellt. 
Bei einer Eingangsspannung von $0.33$ V wurde bei einer zehnfachen Verstärkung der Maximalwert von
$1.9$ V gemessen. Die Verstärkung scheint also zwischen fünf- und sechsfach zu sein. Möglich ist 
auch, dass das exakte Maximum einfach nicht getroffen wurde, da der Sinusgenerator sich recht 
eigenwillig verhielt. Aufgrund eines Wackelkontaktes musste nach der Gütebestimmung der 
Selektivverstärker gewechselt werden, sodass nicht sicher ist, ob die zuvor berechnete Güte und
auch die Verstärkung durch den Selektivverstärker sich so verhält, wie zuvor bestimmt.\\
Weitere Fehlerquellen sind der Innenwiderstand des Messgerätes und die Temperaturabhängigkeit der 
Suszibilität.

