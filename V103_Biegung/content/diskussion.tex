\section{Diskussion}
\label{sec:Diskussion}

Die Abweichungen von den Literaturwerten werden im Folgenden mithilfe der Formel
\begin{equation*}
    \Delta x = \frac{x_{\symup{theoretisch}} - x_{\symup{gemessen}}}{x_{\symup{theoretisch}}}
\end{equation*}
bestimmt. \\
\\
\noindent Die Messwerte, Theoriewerte und prozentualen Abweichungen werden in Tabelle \ref{tab:vergleich}
zusammengesetzt. Dabei stammt der Theoriewert für Messing aus Quelle \cite{messing} und der für Eisen aus Quelle \cite{eisen}. Zu 
beachten ist, dass es bei Messing als Legierung stark auf die Zusammensetzung ankommt und es daher eine große 
Toleranz für das Elastizitätsmodul gibt.

\begin{table}
    \centering
    \caption{Literaturwerte, Messwerte und zugehörige prozentuale Abweichungen zur Bestimmung der Elastizitätsmodule.}
    \label{tab:vergleich}
    \begin{tabular}{c c c c}
        \toprule
        Messung & Ergebnis $\mathbin{/} \si{\newton\per\m\squared}$ & Theoriewert $\mathbin{/} \si{\newton\per\m\squared}$ & Abweichung $\%$\\
        \midrule
        $E_{\symup{Messing}}$, einseitige Befestigung & $\left( 0.945 \pm 0.035 \right)\cdot 10^{11}$ & $\left(1 \pm 0.22 \right) \cdot 10^{11}$ & $5 \pm 19$ \\
        $E_{\symup{Eisen}}$, einseitige Befestigung & $\left( 1.761 \pm 0.014\right)\cdot 10^{11}$ & $1.96 \cdot 10^{11} $ & $10.2 \pm 0.7$\\
        $E_{\symup{Eisen}}$, beidseitig, $x \leq L/2$ & $\left( 2.030 \pm 0.340 \right) \cdot 10^{11}$ & $1.96 \cdot 10^{11} $ & $4 \pm 18$ \\
        $E_{\symup{Eisen}}$, beidseitig, $x \geq L/2$ & $\left( 2.410 \pm 0.230 \right) \cdot 10^{11}$ &  $1.96 \cdot 10^{11} $ & $23 \pm 12$ \\
        \bottomrule
    \end{tabular}
\end{table}

\noindent Die Werte sind teilweise von großen Unsicherheiten betroffen, woraus sich schließen lässt, dass einige Messfehler vorliegen. Dazu gehören die Ungenauigkeit der Messuhren,
die sich nicht fest auf einen Wert einstellen ließen sondern stets schwankten, hinzu kommt, dass die Messuhren sich bei Erschütterungen
des Tisches auf dem die Versuchsanordnung aufgebaut war sprunghaft verstellten. Da es bei der Messung auf die Differenz zwischen Biegung mit Gewicht und ohne Gewicht 
ankommt, mussten die Gewichte bei jeder Messung abgenommen und wieder aufgesetzt werden, wodurch Erschütterungen nicht verhindert werden konnten.\\
Eine weitere Schwierigkeit ist, dass die Messuhren nicht perfekt auf dem runden Stab aufliegen können, sodass kleine Schwankungen durch die unterschiedlichen Positionen der 
Messuhren auf dem runden Stab hinzukommen.\\
Zudem kann trotz großer Sorgfalt bei der Durchführung nicht davon ausgegangen werden, dass das Gewicht bei jedem Messwert auf genau dieselbe Stelle 
gesetzt wurde, sodass auch hier eine Unsicherheit hinzu kommt. \\
Die größte Abweichung vom Theoriewert liegt bei der Berechnung des Elastizitätsmoduls aus der beidseitigen Messung im Bereich $x \geq L/2$ vor. 
Dabei ist zu Bedenken, dass bei der beidseitigen Messung nur die linke Seite des Stabes eingespannt war und die andere lediglich auf einer Schiene auflag. Sie war also 
möglicherweise anfälliger für Erschütterungen und hat daher eine höhere Unsicherheit.\\