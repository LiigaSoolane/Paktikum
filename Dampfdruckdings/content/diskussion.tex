\section{Diskussion}
\label{sec:Diskussion}
\subsection{Erste Versuchsreihe: Niedrigdruckbereich}
Die Messwerte lauten:
\begin{align*}

\end{align*}
%Literaturwerte?
Die Literaturwerte für diese Messung lauten 
\begin{align*}

\end{align*}
Zunächst wurde es versäumt, die Apparatur korrekt einzuschalten, was zu einem langsamen Temperaturanstieg über einen längeren Zeitraum geführt hat. Als der Fehler aufgefallen
ist musste gewartet werden, bis die Substanz wieder auf Zimmertemperatur abkühlte. Aus Zeitgründen konnte das nicht gänzlich realisiert werden, weshalb die Starttemperatur 
mit $ \si{degreeCelsius}$ deutlich höher ausfiel als in der Versuchsanleitung beabsichtigt. Es ist zu erwarten, dass sich so ein negativer Einfluss auf die restliche Messung
ergeben hat. \\
Auch der Fehler, der beim Ablesen von Thermo- oder Manometer entsteht, ist erwähnenswert. Da das Manometer ein Digitales war, ist der erwartete Messfehler für den ablegesenen 
Druck gering, allerdings wurde ein analoges Thermometer eingesetzt, was unweigerlich einen Messfehler mit sich bringt.
%Werte, die eine Sonderbehandlung brauchen?

\subsection{Zweite Versuchsreihe: Hochdruckbereich}
Für diesen Bereich wurde gemessen:
\begin{align*}

\end{align*}
%Literaturwerte?
\begin{align*}

\end{align*}
Wie bereits beschrieben begann die Apparatur bei einem Druck von $\symup{p} = \si{\bar}$ zu zischen, was verbunden mit dem beobachteten Druckabfall den Schluss nahelegt, dass
Gas aus dem Versuchsaufbau entwich. Dadurch konnte die Messung nicht in Gänze durchegeführt werden und wurde nach etwa der Hälfte abgebrochen, wodurch ein Teil der für die 
vollständige Darstellung nötigen Werte fehlen. Zu Bedenken ist auch, dass die mangelhafte Dichtung des Gerätes auch die vorher gemessenen Werte negativ beeinflusst haben 
könnte, da kein abgeschlossenes System mehr vorlag und so auch kein reversibler Prozess, was allerdings eine Grundannahme in der theoretischen Herleitung der Gleichung war. \\
Hinzu kommt der typische menschliche Fehler, der aus dem Ablesen an analogen Thermometern und Manometern folgt. Es ist zudem zu bemerken, dass die Nadel des Manometers bei 
Erschütterung des Untergrunds sprunghaft um wenige Millibar anstieg, sodass die abgelesenen Werte möglicherweise keinen gleichmäßigen Abstand zueinander haben. \\
Des Weiteren sind in die Herleitung der Differentialgleichung, nach der die Kurve bestimmt wurde, viele Näherungen eingegangen, die eine Abweichung von tatsächlichen Werten
zur Folge haben.
%irgendwelche Werte die gesondert diskutiert werden sollten?