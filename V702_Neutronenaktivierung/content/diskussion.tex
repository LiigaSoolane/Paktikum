\section{Diskussion}
\label{sec:Diskussion}

\begin{table}[H]
 \centering
 \caption{Theoriewerte und prozentuale Abweichungen.}
 \label{tab:theo}
 \begin{tabular}{lccc}
  \toprule
  Wert & Theoriewert & Messung & Abweichung in Prozent \\
  \midrule
  Halbwertszeit von Vanadium & $224.4 \si{\s}$ & $215 \pm 7 \si{\s}$ & $10$ \\
  Genauer gemesse Halbwertszeit von Va & $224.4 \si{\s}$ & $208 \pm 13 \si{\s}$ & $17$ \\
  Halbwertszeit von Rh 104 & $42.3 \si{\s}$ & $43.05 \pm 0.001 \si{\s}$ & $1.8$ \\
  Halbwertszeit von Rh 104i & $260 \si{\s}$ & $258.15 \pm 0.001 \si{\s}$ & $0.75$ \\
  \bottomrule
 \end{tabular}
\end{table}

\subsection{Vanadium}

In Tabelle \ref{tab:theo} werden die Theorie-  und gemessenen Werte verglichen.
Ungewöhnlich ist dabei, dass die Abweichung, die bei der prinzipiell genaueren Methode herauskommt,
größer ist als bei der prinzipiell ungenaueren Methode. 
Der größere Fehler des Wertes lässt sich dadurch erklären, dass durch eine geringere Anzahl an Werten auch ein größerer statistischer Fehler entsteht. 
Außerdem ist die generelle Unsicherheit des Versuches recht hoch, beispielsweise durch die Auswirkungen einer möglicherweise nicht idealen Wahl von $\increment t$
für die Messung. Durch ein zu kleines Zeitintervall kann ein statistischer Fehler dazu kommen, während ein zu großes Zeitintervall zu einem systematischen Fehler in 
$\lambda$ führen kann. \\
Zudem ist mit dem Zählrohr, das für die Messung des Zerfalls genutzt wurde, keine einwandfreie Messung möglich, da nach jeder Detektion für eine Totzeit keine Werte 
aufgenommen werden können. Dieser Effekt kommt zu Anfang de Messung mehr zum Tragen, da dann mehr Impulse in einem Zeitintervall aufgenommen werden.\\
Eine Verbesserung der Werte wäre außerdem durch eine längere Messung der Untergrundstrahlung möglich, um den statistischen Fehler zu verringern, oder durch bessere Abschirmung
des Geiger-Müller-Zählrohres.

\subsection{Rhodium}

Auch für Rhodium werden die Messwerte in Tabelle \ref{tab:theo} mit den Theoriewerten verglichen.
In Abbildung \ref{fig:regress}, die die entsprechenden Kurven darstellt, ist die Güte dieser 
Werte auch erkennbar, da die zusammengesetzte Kurve genau durch die Datenpunkte verläuft. \\
Geringe Abweichungen bei den einzelnen Kuven für die jeweiligen Isotope kommen dadurch zustande, dass der Einfluss des andern Isotops bei der realen Messung natürlich auch noch
einen Einfluss hat, sodass die Werte nicht perfekt repräsentiert sind. Dies ist erst bei Kombination der Kurven der Fall. 