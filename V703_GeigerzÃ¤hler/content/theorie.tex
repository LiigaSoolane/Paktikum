\section{Theorie}
\label{sec:Theorie}

\sucsection{Wirkungsweise}
\begin{figure}
 \centering
 \includegraphics[width=textwidth]{}
 \label{fig:aufn}
\end{figure}

Beim Geiger-Müller-Zählrohr umgibt ein Stahlmantel in zylindrischer Form
ein Gasgemisch, meist mit Alkoholanteil. Innen befindet sich ein Anodendraht,
dessen Gegenstück der Zylinder darstellt, der also auch als Kathode fungiert.\\
Geladene Teilchen tendieren dazu, stak zu reagieren, daher ist die wahrscheinlichkeit,
dass sie in das Innere des Zählrohrs eintreten, ausgesprochen gering. 
Stattdessen werden sie im metallischen Mantel absorbiert. Um dafür zu 
sorgen, dass auch einige Teilchen ins Innere gelangen, wird an die Seite,
die nicht mit dem Strom verbunden ist, eine dünne Wand aus Material mit Atomen
niedriger Ordnungszahl. Häufig wird beispielsweise Mylar-Folie verwendet. 
Diese Zählrohre werden auch Endfensterzählrohre genannt. Durch den im Zylinder
herrschenden Unterdruck beugt sich die Folie nach Innen, wie in \ref{fig:aufn} 
zu sehen ist.\\
Durch den zylindersymmetrischen Aufbau entsteht bei Anlegen einer äußeren Spannung
ein radialsymmetrisches Feld, für das gilt
\begin{equation}
\symup{E(r) }= \frac{U}{r ln\left(\frac{r_k}{r_a} \right) } 
\end{equation}
Wobei $r_a$ den Radius der Anode und $r_k$ den Radius der Kathode bezeichnen.
Ein geladenes Teilchen in diesem Feld wird um einen Wert beschleunigt, der
sich in einem Bereich von $\left( r_a < r < r_k \right)$ proportional zu $\frac{1}{r}$ 
verhält. \\
\\
Dringt ein geladenes Teilchen in das Zählrohr ein, wird seine Energie 
sich nach und nach durch Ionisation aufbrauchen, bis es vollständig absorbiert
ist. Die Energie, die das Teilchen zu Anfang hat, ist aber um ein Vielfaches
höher als die Energie pro gebildetem Ionenpaar, wodurch die Anzahl an 
entstehenden Elektronen und positiven Ionen proportional zur Anfangsenergie
ist. \\
Nach dieser Primärionisation laufen je nach angelegter Spannung verschiedene
Vorgänge ab. In Abbildung \ref{fig:hä} sind die verschiedenen möglichen
Situationen dargestellt.\\
Ist die Spannung gering, so erreichen nicht alle erzeugten Elektronen den Anodendraht
und die Anderen unterliegen vorher der Rekombination. In der Abbildung ist
dies der Bereich 1.\\
Als Ionisationskammer wird ein Gerät bezeichnet, das mit etwas höheren Spannungen
arbeitet, sodass die Rekombinationswahrscheinlichkeit weitaus geringer ist.
Es gelangen praktisch alle Elektronen zum Anodendraht und es fließt ein kontinuierlicher
Ionisationsstrom zwischen Kathode und Anode, der proportional zu Energie und
Intensität der einfallenden Strahlung ist. Bei weniger hohen Strahlintensitäten
ist der Einsatz einer Ionisationskammer nicht sinnvoll, da die entstehenden Ströme 
sehr klein sind.\\
Wenn die Spannung erhöht wird, steigt die Feldstärke in Drahtnähe, sodass die 
freigesetzten Elektronen ionisieren können, da sie zwischen den Zusammenstößen 
mit Argon-Atomen ausreichend Energie aufnehmen. Die durch diesen Stoßionisation genannten
Vorgang entstandenen Elektronen können auf dieselbe Weise wieder ionisieren,
solange die Spannung dazu aureicht, es bildet sich also eine sogenannte Townsend-Lawine
aus. \\
Pro einfallendem Teilchen sammelt sich nun so viel Ladung, dass ein Ladungsimpuls 
gemessen werden kann. Aufgrund des Proportionalitätszusammenhangs zwischen
gesammelter Ladung und Anfangsenergie ist dieser Ladungsimpuls ein Maß für 
die Teilchenenergie. Es wird in diesem Spannungsbereich daher von einem 
Proportionalzählrohr gesprochen.\\
Oberhalb dieses Proportionalbereichs ist die gesammelte Ladung unabhängig 
von der Anfangsenergie, allein das Volumen des Zählrohres und die Spannung sind
dann entscheidende Parameter. Durch die Elektronenlawinen werden Argon-Atome angeregt
und emittieren ultraviolette Photonen, wodurch dann wiederum 
freie Elektronen entstehen, aufgrund der Neutralität der Photonen im gesamten Volumen.
Es ist dann wieder nur die Intensität einfallender Strahlung messbar, nicht mehr
die Energie, allerdings arbeitet das Geiger-Müller-Zählrohr sehr viel effektiver als
das Proportionalzählrohr, da es einen geringeren elektronischen Aufwand verlangt.
Auch ist der Ladungsimpuls unabhängig vom Ionisationsvermögen der einfallenden Strahlung.

\subsection{Totzeit, Nachentladungen}

\subsection{Charakteristik}

\subsection{Ansprechvermögen}
