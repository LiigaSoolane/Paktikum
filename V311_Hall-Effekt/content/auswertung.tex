\section{Auswertung}
\label{sec:Auswertung}

\subsection{Messung des Widerstandes}

Zuerst wurde der Widerstand von Kupfer bestimmt. Dazu wurde eine Probe mit 
einem Kupferdraht der Länge $l = 1.37 \si{\m}$ und der Dicke $d = 0.00037 \si{\m}$
verwendet. \\
Es wurde gemessen, dass der Widerstand $R = 2.65 \si{\ohm}$ beträgt.

\subsection{Messung des Magnetfeldes}

Als Nächstes wurde mithilfe einer kleinen Hall-Sonde das von den Spulen erzeugte Magnetfeld gemessen, dazu wurde der Strom, der in die Spulen geleitet wurde, stückweise erhöht, 
nach Umpolung der Stromzufuhr wurde die Messung wiederholt. Es ergaben sich die in Tabelle \ref{tab:mag} dargestellten Messdaten.

\begin{table}
 \centering
 \caption{Messung des Magnetfeldes.}
 \label{tab:mag}
 \sisetup{table-format=2.1}
 \begin{tabular}{S[table-format=3.1]S S S S}
  \toprule
  & \multicolumn{2}{c}{Magnetfeld $B \mathbin{/} \si{\tesla}$}\\
  \cmidrule(lr){2-3}
  {$I \mathbin{/} \si{\ampere}$} & {ansteigend} & {abfallend}\\
  \midrule
  0     &  0.0043  &  0.0034  \\
  0.5   &  0.1083  &  0.0991  \\  
  1     &  0.2175  &  0.2032 \\
  1.5   &  0.3224  &  0.3076  \\ 
  2     &  0.422   &  0.4105 \\
  2.5   &  0.5202  &  0.5143  \\ 
  3     &  0.6179  &  0.616  \\
  3.5   &  0.7172  &  0.7121  \\       
  4     &  0.8396  &  0.7936 \\
  4.5   &  0.9292  &  0.8527  \\ 
  4.741 &  0.9624  &  0.8732  \\  
  \bottomrule
 \end{tabular}
\end{table} 

\noindent Beide Verläufe wurden mithilfe linearer Regression in kontinuierliche Verläufe der Form $y = a + b*x$ überführt, durch Rechnung mit Scientific Python ergeben sich für das 
Magnetfeld vor der Umpolung
\begin{align*}
a = 0.204 \pm 0.002\\
b = 0.011 \pm 0.005\\
B_{\text{ansteigend}} = (0.204 \pm 0.002) + (0.011 \pm 0.005) \cdot I
\end{align*}

und für das Magnetfeld nach Umpolung

\begin{align*}
a = 0.19 \pm 0.005\\
b = 0.02 \pm 0.014\\
B_{\text{abfallend}} = (0.19 \pm 0.005) + (0.02 \pm 0.014) \cdot I
\end{align*}

\noindent Die Abbildung \ref{fig:plots} zeigt das jeweilige Magnetfeld aufgetragen gegen die Stromstärke. 

\begin{figure}
 \centering
 \includegraphics[width=\textwidth]{mag.pdf}
 \caption{Plots der Fit-Funktionen und Messwerte.}
 \label{fig:plots}
\end{figure} 
\FloatBarrier
\noindent Um genauere Werte zum Weiterrechnen zu erzielen, wurde der Mittelwert aus
beiden Geraden verwendet. Dann folgt eine Relation von Magnetfeld und Stromstärke nach

\begin{align*}
a = 0.197 \pm 0.002\\
b = 0.014 \pm 0.007\\
B_{\text{mittel}} = (0.197 \pm 0.002) \pm (0.014 \pm 0.007) \cdot I
\end{align*}


\subsection{Berechnung der Hall-Spannung}
Mithilfe einer Kupferprobe, durch die ein Strom geleitet wurde, wurde die Hall-Spannung sowohl bei konstant gehaltenem Spulenstrom als auch 
bei konstant gehaltenem Probenstrom. Ab hier wird ein Subskript Pk dafür stehen, dass es sich um den Wert für die Messreihe mit konstantem 
Probenstrom handelt, während ein Subskript Sk für einen konstanten Spulenstrom steht.\\
In Tabelle \ref{tab:SKhall} sind die Werte für konstanten Spulenstrom dargestellt, während Tabelle \ref{tab:PKhall} die Messdaten für einen 
konstanten Probenstrom zeigt.
\begin{table}
 \centering
 \caption{Messung der Spannung bei konstantem Probenstrom.}
 \label{tab:PKhall}
 \sisetup{table-format=2.1}
 \begin{tabular}{S[table-format=3.1]S S S S}
  \toprule
  & \multicolumn{2}{c}{Spannung $U_\text{Pk} \mathbin{/} \si{\milli\volt}$}\\
  \cmidrule(lr){2-3}
  {$I \mathbin{/} \si{\ampere}$} & {ansteigend} & {abfallend}\\
  \midrule
  0     & 0.002 & 0.003  \\
  0.5   & 0.002 & 0.003  \\  
  1     & 0.003 & 0.002 \\
  1.5   & 0.004 & 0.002  \\ 
  2     & 0.004 & 0.002 \\
  2.5   & 0.005 & 0.001  \\ 
  3     & 0.006 & 0.001 \\
  3.5   & 0.006 & 0.001  \\       
  4     & 0.006 & 0 \\
  4.5   & 0.007 & 0  \\ 
  5     & 0.007 & 0  \\  
  \bottomrule
 \end{tabular}
\end{table} 

\begin{table}
 \centering
 \caption{Messung der Spannung bei konstantem Spulenstrom.}
 \label{tab:SKhall}
 \sisetup{table-format=2.1}
 \begin{tabular}{S[table-format=3.1]S S S S}
  \toprule
  & \multicolumn{2}{c}{Spannung $U_\text{Sk} \mathbin{/} \si{\milli\volt}$}\\
  \cmidrule(lr){2-3}
  {$I \mathbin{/} \si{\ampere}$} & {ansteigend} & {abfallend}\\
  \midrule
  0     & 0.003  & 0.004  \\
  0.5   & 0.003  & 0.003  \\  
  1     & 0.003  & 0.002  \\
  1.5   & 0.004  & 0.002  \\ 
  2     & 0.004  & 0.002  \\
  2.5   & 0.004  & 0.002  \\ 
  3     & 0.005  & 0.002  \\
  3.5   & 0.005  & 0.001  \\       
  4     & 0.005  & 0.001  \\
  4.5   & 0.005  & 0  \\ 
  4.741 & 0.006  & 0  \\  
  \bottomrule
 \end{tabular}
\end{table} 

\noindent Da die gemessene Spannung nicht der Hall-Spannung entspricht, sondern noch ein Störpotential
beinhaltet, wird nach
\begin{equation}
 U_H = \frac{1}{2} \left( U_\text{aufsteigend} - U_\text{abfallend} \right)
 \label{eqn:hallformel}
\end{equation}
die Hall-Spannung berechnet. So ergeben sich die in Tabelle \ref{tab:hallsp} dargestellten Werte.

\begin{table}
 \centering
 \caption{Nach \eqref{eqn:hallformel} berechnete Hall-Spannung.}
 \label{tab:hallsp}
 \sisetup{table-format=2.1}
 \begin{tabular}{S[table-format=3.1]S S S S}
  \toprule
  & \multicolumn{2}{c}{Hall-Spannung $U_H \mathbin{/} \si{\volt}$}\\
  \cmidrule(lr){2-3}
  {$I \mathbin{/} \si{\ampere}$} & {konstater Spulenstrom} & {konstater Probenstrom}\\
  \midrule
  0     & -5   & -5   \\
  0.5   &  0.1 & -5   \\  
  1     &  5   &  5   \\
  1.5   &  0.1 &  0.1 \\ 
  2     &  0.1 &  0.1 \\
  2.5   &  0.1 &  0.2 \\ 
  3     &  0.15&  0.25\\
  3.5   &  0.2 &  0.25\\       
  4     &  0.2 &  0.3 \\
  4.5   &  0.25&  0.35\\ 
  5     &  0.3 &  0.35\\  
  \bottomrule
 \end{tabular}
\end{table} 

\noindent 

\subsection{Berechnung der mikroskopischen Leitfähigkeitsparameter}

Mit diesen Werten lassen sich die mikroskopischen Leitfähigkeitsparameter bestimmen, die das verhalten eines Metalles charakterisieren.

\subsubsection{Ladungsträgerdichte}
In der Theorie wurde Formel \eqref{eqn:hallspannung} für die Hall-Spannung hergeleitet.
Diese lässt sich mit Leichtigkeit nach $n$ umstellen, sodass sich ergibt
\begin{equation*}
n = \frac{-1}{U_H e_0} \frac{B I_\text{quer}}{d}
\end{equation*}

Dabei ist $e_0 = -1.601 \cdot 10^(-19)\si{\coulomb}$ die Elementarladung (Quelle \cite{e0}), 
$d = 0.032 \si{\milli\meter}$ die Dicke der Kupferprobe
und $I_\text{quer}$ der Probenstrom.
Es folgen also auch wieder zwei Werte für n, einer für konstanten Probenstrom und einer für
konstanten Spulenstrom. Der jeweils konstante Strom wurde dabei auf $2 \si{\ampere}$ gehalten.\\
Nach Einsetzen der Werte folgen die Daten in Tabellen \ref{tab:n1} und \ref{tab:n2}.

\begin{table}
 \centering
 \caption{Ladungsträgerdichte bei konstantem Spulenstrom.}
 \label{tab:n1}
  \begin{tabular}{
      S[table-format=1.3]
      @{${}\pm{}$}
      S[table-format=1.3]
  }
   \toprule
   \multicolumn{2}{c}{$n_\text{Sk} \cdot 10^{29} \mathbin{/} \si{\per\m\cubed}$} \\
   \midrule
    -0.119 & 0.056 \\
    0.443 & 0.027 \\
    1.655 & 0.058 \\
    1.211 & 0.031 \\
    1.594 & 0.038 \\
    1.979 & 0.037 \\
    1.574 & 0.027 \\
    1.372 & 0.022 \\      
    1.564 & 0.024 \\
    1.406 & 0.021 \\
    1.298 & 0.019 \\
   \bottomrule
 \end{tabular}
\end{table}

\begin{table}
 \centering
 \caption{Ladungsträgerdichte bei konstantem Probenstrom.}
 \label{tab:n2}
 \begin{tabular}{
     S[table-format=1.3]
     @{${}\pm{}$}
     S[table-format=1.2]
 }
  \toprule
  \multicolumn{2}{c}{$n_\text{Pk} \cdot 10^{29} \mathbin{/} \si{\per\m\cubed}$}\\
  \midrule
  0.0001 & 0.038 \\
  -0.797 & 0.017\\
  1.594 & 0.038\\
  1.195 & 0.024\\
  1.594 & 0.038\\
  0.969 & 0.021\\
  0.956 & 0.019\\
  1.117 & 0.024\\
  1.062 & 0.023\\
  1.024 & 0.022\\
  1.138 & 0.041\\
  \bottomrule
 \end{tabular}
\end{table}

\noindent Aus den jeweiligen Datenreihen wurden die Mittelwerte bestimmt, sie ergaben sich zu
\begin{align*}
n_\text{Sk} = \left( 1.271 \pm 0.022 \right) \cdot 10^{29} \si{\per\m\cubed}\\
n_\text{Pk} = \left( 0.899\pm 0.19 \right) \cdot 10^{29} \si{\per\m\cubed}
\end{align*}

Der Fehler wurde dabei aus mit den Fehlern der linearen Regression nach der Gauss'schen Fehlerrechnung
\begin{equation}
\sigma_f = \sqrt{\sum_{i=0}^{N} {\frac{\partial f}{\partial x_i} \cdot \sigma_{x_i}}}
\label{eqn:gauss}
\end{equation}
mithilfe von Numeric Python berechnet.

\subsubsection{Anzahl Ladungsträger pro Atom}

Um die Anzahl der Ladungsträger pro Atom herauszufinden, muss die Ladungsträgerdichte durch 
die Atomdichte von Kupfer geteilt werden.\\
Diese ergibt sich aus $a = \frac{N}{V}$, wobei N die Gesamtzahl der Atome in Kupfer sind und V das Volumen. 
Allerdings ist auch 
\begin{equation*}
\rho = \frac{N m_\text{Cu}}{V}
\end{equation*}
Daraus folgt
\begin{equation*}
a = \frac{\rho}{m_\text{Cu}} \\
\end{equation*}
Dabei bezeichnet $m_\text{Cu}$ die Masse eines Kupferatoms $m_\text{Cu} = 63.4 \si{\atomicmassunit}$ (Quelle \cite{beweglichkeit}), mit $u = 1.661 \cdot 10^{-27}\si{\kilo\g}$ (\cite{beweglichkeit}).
So ist
\begin{equation}
z = \frac{n}{a} = \frac{n m_\text{Cu}}{\rho}
\end{equation}
Nach Einsetzen der jeweiligen Werte folgen
\begin{align*}
z_\text{Sk} = 1.501 \pm 0.026\\
z_\text{Pk} = 1.061 \pm 0.023
\end{align*}

Die Fehler wurden wieder automatisch durch Python mithilfe der Gauss'schen Fehlerrechnung 
\eqref{eqn:gauss} errechnet.

\subsubsection{Mittlere Driftgeschwindigkeit}

Die mittlere Driftgeschwindigkeit wurde mithilfe der in der Theorie hergeleiteten
Formel \eqref{eqn:stromdichte} für die Stromdichte j  berechnet. Nach Umstellen nach $v_d$ ergibt sich
\begin{equation*}
\bar{v_d} = - \frac{n \cdot e_0}{j} 
\end{equation*}
Durch Berechnen mit den Werten von $n$ und der Vorgabe $j = 1 \si{\ampere\per\milli\meter\squared}$ (\cite{V311})
folgt:
\begin{align*}
\bar{v}_\text{d,Sk} = \left( 4.91 \pm 0.09 \right) \cdot 10^{-5} \si{\m\per\s}\\
\bar{v}_\text{d,Pk} = \left( 6.95 \pm 0.15 \right) \cdot 10^{-5} \si{\m\per\s}
\end{align*}

\subsubsection{Mittlere Flugzeit}

Für den Widerstand wurde in der Theorie die Formel \eqref{eqn:widerstand} hergeleitet. Da der Widerstand
von Kupfer im Rahmen des Versuchs gemessen wurde, lässt sich diese Formel nach
der mittleren Flugzeit $\bar{\tau}$ umstellen.
\begin{equation*}
\bar{\tau} = \frac{2 m_0 L}{R e_0^2 n Q}
\end{equation*}
Dabei bezeichnet $Q$ den Querschnitt des Drahtes, mit dessen Hilfe der Widerstand gemessen wurde und 
$m_0 = 9.108 \cdot 10^{-31}\si{\kg}$ ist die Ruhemasse eines Elektrons (Quelle \cite{m0}).
Er ist mit der Formel
\begin{equation}
Q = \pi \left( \frac{d}{2} \right)^2
\end{equation}
zu Berechnen. So folgt für $\tau$
\begin{align*}
\tau_\text{Sk} = \left( 3.59 \pm 0.06 \right) \cdot 10^{-13} \si{\s}\\
\tau_\text{Pk} = \left( 5.08 \pm 0.11 \right) \cdot 10^{-13} \si{\s}
\end{align*}

\subsubsection{Beweglichkeit}

Wie in der Theorie beschrieben ist die Beweglichkeit die Proportionalitätskonstante zwischen 
der mittleren Driftgeschwindigkeit und der äußeren Feldstärke. Für sie gilt (\eqref{eqn:beweg})
\begin{align*}
\mu = -\frac{e_0}{m_0} \cdot \bar{\tau} \\
\implies \mu_\text{Sk} = 0.0316 \pm 0.0006 \si{\m\squared\per\V\s} \\
\mu_\text{Pk} = 0.0446 \pm 0.0009 \si{\m\squared\per\V\s}
\end{align*}

\subsubsection{Totalgeschwindigkeit}

Die Totalgeschwindigkeit wurde mit der Gleichung \eqref{eqn:totv} berechnet.
Dafür musste zunächst die Fermi-Energie $E_f$ berechnet werden. Für sie gilt (nach \eqref{eqn:fermi})
\begin{align*}
E_f = \frac{h^2}{2 m_0} \left( \left( \frac{3}{8 \pi} \right)^2 \right)
\end{align*}
Mit dem Plankschen Wirkungsquantum $h = 6.625 \cdot 10^{-34}\si{\joule\s}$ (Quelle \cite{Planck}).
Es sind also $\bar{v} = \left( 1.801 \pm 0.11 \right) \cdot 10^{6} \si{\m\per\s}$ und 
$\bar{v} = \left( 1.604 \pm 0.12 \right) \cdot 10^{6} \si{\m\per\s}$.

\subsubsection{Mittlere freie Weglänge}

Die mittlere freie Weglänge ist die Strecke, die in der mittleren Flugzeit
zurückgelegt wird. Es ist also anschaulich, dass (auch in der Theorie hergeleitet \eqref{eqn: Weglaenge}) 
\begin{equation*}
\bar{l} = \bar{ \tau } \bar{v}
\end{equation*}
Nach Einsetzen der Werte folgt so $\bar{l_\text{Sk}} = \left( 6.47 \pm 0.08 \right) \cdot 10^{-7} \si{\m}$
und $\bar{l_\text{Pk}} = \left( 8.15 \pm 0.12 \right) \cdot 10^{-7} \si{\m}$.
