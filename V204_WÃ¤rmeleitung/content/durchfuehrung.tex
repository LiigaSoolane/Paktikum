\section{Durchführung}
\label{sec:Durchführung}
An einer strombetriebenen Wärmequelle sind einseitig vier Metallstäbe aus verschiedenen 
Materialien, Edelstahl, Aluminium, ein schmaler Messingstab und ein breiter Messingstab, 
angebracht. An jeweils zwei, voneinander 3 cm entfernten Stellen der vier Metallproben sind 
Temperaturmessgeräte installiert. Das Heizelement wird über einen Schalter auf \textit{Heat}
oder \textit{Cool} geregelt. Die Daten Temperaturmessgeräte werden
mithilfe eines Datenloggers aufgenommen.


\subsection{Statische Methode}
    In diesem Versuchteil wird das Heizelement
    durch den konstanten Strom von 5 V beheizt, wobei eine Abstastrate von $\Delta s= 10$ s verwendet wird. Der Temperaturverlauf
    wird über ein Zeitintervall von 1000 s 
    verfolgt. Die Daten werden mit einem Datenlogger simultan aufgenommen. 
    Durch den aufgenommenen Temperaturverlauf wird die Wärmeleitfähigkeit bestimmt.

\subsection{Dynamische Methode}
    Als dynamische Methode wird in diesem Versuch das Angström-Messverfahren verwendet. Dabei werden
    die zu untersuchende Stab über eine Periode von 80 s abwechselnd geheizt und gekühlt, Daten werden mit einer Abtastrate von $\Delta s = 2s$ aufgenommen.
    Die Wärmequelle wird nun über eine Spannung von 8 V betrieben.\\
    Die erste Messreihe dieser Art wird mit einer Periodendauer von 80 s über 
    10 Perioden durchgeführt.\\
    In einer weiteren Messreihe wird eine Periodendauer von 200 s über 6 Perioden verwendet.
