\section{Theorie}
\label{sec:Theorie}

\subsection{Zielsetzung}
    Ziel dieses Versuchs ist es, die Bewegung der Leitungselektronen eines metallischen Stoffes
    zu untersuchen. Diese sind die Elektronen auf dem höchsten Energieband, welches auch 
    Leitfähigkeitsband oder Valenzband genannt wird. 
\subsection{Berechnung der Leitfähigkeit eines Metalles}
    Das Verhalten der Leitungselektronen kann durch das Verhalten von Atomen in einem
    idealen Gas angenähert werden. Wie die Atome eines idealen Gases führen auch 
    die Leitungselektronen untergeordnete Bewegungen aus. Dabei stoßen sie 
    immer wieder mit Strukturdefekten zusammen.
    Um die mittlere Flugzeit $\overline \tau$ eines Leitungselektrons zu bestimmen, 
    wird über die Zeitintervalle zwischen zwei Zusammenstößen gemittelt.\\
    Die mittlere Driftgeschwindigkeit 
    in Richtung eines elektrischen Feldes ergibt sich aus den getroffenen Annahmen zu 
    \begin{equation}
        \label{eqn:driftv}
        \vec{\overline v_d} = \dfrac{1}{2} \Delta \vec{\overline v} = -\dfrac{e_0}{2 m_0} \vec E \overline \tau
    \end{equation}
    Die Stromdichte beträgt dann
    \begin{equation}
        \label{eqn:stromdichte}
        j = - n\ \overline v_d \ e_0 = \dfrac{e_0²}{2 m_0}n\ \overline \tau \ E
    \end{equation}
    wobei n die Anzahl der Leitungselektronen pro Volumeneinheit ist. Wird davon ausgegangen, 
    dass die Leiterplatte homogen ist, so folgt für den Strom 
    \begin{equation}
        \label{eqn:strom}
        I=\dfrac{e_0²}{2 m_0}n\ \overline \tau \dfrac{Q}{L}U
    \end{equation}
    Diese Gleichung weist die Gestalt des Ohmschen Gestzes $I=\dfrac{U}{R}$ auf. 
    Der Proportionalitätsfaktor $\dfrac{1}{R}$ wird auch elektrische Leitfähigkeit S 
    genannt. Nach Gleichung \eqref{eqn:strom} werden der elektrischer Widerstand 
    und die elektrische Leitfähigkeit wie folgt berechnet:
    \begin{equation}
        \label{eqn:leitfähigkeit}
        S=\dfrac{e_0²}{2 m_0}n\ \overline \tau \dfrac{Q}{L}
    \end{equation}
    \begin{equation}
        \label{eqn:widerstand}
        R=\dfrac{2m_0}{e_0²} \dfrac{1}{n \overline \tau} \dfrac{L}{Q}
    \end{equation}
    Der Widerstand und somit auch die elektrische Leitfähigkeit eines Drahtes
    lassen sich einfach und präzise messen. Sind S und R bekannt, so können 
    die spezifische Leitfähigkeit $\omega$ und der spezifische Widerstand $\rho$ 
    eines Materials durch folgende Gleichungen bestimmt werden. 
    \begin{align}
        \omega= \dfrac{e_0²}{m_0}n\ \overline \tau \\
        \rho = \dfrac{2m_0}{e_0²} \dfrac{1}{n \overline \tau}
    \end{align}
    Außerdem können aus dem Widerstand R nach der Gleichung \eqref{eqn:widerstand} 
    Rückschlüsse über die materialabhängigen mikroskopischen Größen gezogen werden.
    Da allerdings zwei unbekannte mikroskopische Größen auftreten (n und $\overline \tau$)
    muss ein weiterer Zusammenhang hergestellt werden.
\subsection{Der Hall-Effekt}
    Der Hall Effekt wird erzeugt, indem, wie in Abbildung \ref{fig:hall}
    dargestellt, an eine homogene Leiterplatte eine elektrische Spannung angelegt wird,
    so dass die Ladungsträger von einem zum anderen Ende laufen. Zusätzlich wird ein 
    Magnetfeld senkrecht zur Stromrichtung angelegt, wodurch die Leitungselektronen
    durch die Lorentzkraft $F_L = e_0 \ \overline v \ B$ in negative y-Richtung
    abgelenkt werden. 
    \begin{figure}[H]
     \includegraphics[width=\linewidth]{hallsonde.png}
     \caption{Aufbau einer Hallsonde}
     \label{fig:hall}
    \end{figure}
    \noindent Es entsteht also ein zusätzliches elektrisches Feld durch die 
    Ladungsansammlung bzw. den Ladungsentzug an den Punkten A und B. Dieses Feld
    wird so groß, dass es die Lorentzkraft genau aufhebt. Es gilt also:
    \begin{equation}
        e_0 \ E_y = e_0\ \overline v_d \ B
    \end{equation} 
    Werden nun die Punkte A und B aus
    Abbildung \ref{fig:hall} mit einem Spannungsmessgerät verbunden, kann die sogenannte
    Hall-Spannung abgegriffen werden. Sie beträgt
    \begin{equation}
        U_H = E_y\cdot b = \overline v_d\ B \cdot b
    \end{equation}
    mit Länge b der stromdurchflossenen Leiterplatte. Unter Verwendung von Gleichung 
    \eqref{eqn:stromdichte} und \eqref{eqn:strom} ergibt sich $\overline v_d = \dfrac{-I_q}
    {b\cdot d\cdot e_0 \cdot n}$ wobei $I_q$ den Querstrom darstellt, sodass
    \begin{equation}
        \label{eqn:hallspannung}
        U_H = -\dfrac{1}{n e_0} \dfrac{B\cdot I_q}{d}.
    \end{equation}
    So kann man den unbekannten Parameter n in Gleichung \eqref{eqn:widerstand} eliminieren. 
    Mit Hilfe der Hallsonde können also die Anzahl von Leitungselektronen pro Volumeneinheit
    n, die mittlere freie Weglänge $\overline \tau$ und bei Kenntnis des fließenden Stroms auch die 
    mittlere Driftgeschwindigkeit $\overline v_d$ bestimmt werden.
\subsection{Mittlere freie Weglänge und mittlere Geschwindigkeit}
    Weiterhin soll die mittlere freie Weglänge $\overline l$ bestimmt werden. Diese 
    kann durch die Gleichung
    \begin{equation}
        \label{eqn: Weglaenge}
        \overline l = \overline \tau \cdot \mid v \mid
    \end{equation}
    bestimmt werden, Dazu muss allerdings die mittlere Geschwindigkeit $\mid v \mid$ der Teilchen
    bekannt sein. Diese lässt sich aus dem Äquipartitionstheorem 
    \begin{equation}
        \vec E_{kin}=\dfrac{3}{2} k T
    \end{equation}
    mit der Temperatur T und Boltzmannkonstante k herleiten. Aus dieser und der Beziehung
    \begin{equation*}
        \overline E_{kin} = \dfrac{m_0}{2}\mid \overline v² \mid
    \end{equation*}
    folgt
    \begin{equation}
        \mid \overline v \mid = \sqrt{\dfrac{3kT}{m_0}}.
    \end{equation}
    Aufgrund des Pauli-Prinzips können die Elektronen aber nicht jede beliebige Geschwindigkeit 
    annehmen. Die Wahrscheinlichkeiten für die unterschiedlichen Energien kann durch die
    Fermi-Dirac-Verteilung berechnet werden. Daher lautet die mittlere Geschwindigkeit
    \begin{equation}
        \label{eqn:totv}
        \mid \overline v \mid \approx \sqrt{\dfrac{2 E_F}{m_0}}
    \end{equation}
    und somit eingesetzt in \eqref{eqn: Weglaenge} die mittlere Weglänge
    \begin{equation}
        \overline l \approx \overline \tau \sqrt{\dfrac{2 E_F}{m_0}}.
    \end{equation}
    Dabei stellt $E_F$ die Fermi-Energie dar, die sich durch
    \begin{equation}
        \label{eqn:fermi}
        E_F = \dfrac{h²}{2 m_0}\sqrt[3]{(\dfrac{3}{8\pi }n)²}
    \end{equation}
    berechnen lässt.
\subsection{Beweglichkeit der Ladungsträger}
    Als Beweglichkeit wird die Proportionalitätskonstante 
    \begin{equation}
        \label{eqn:beweg}
        \mu = \dfrac{\overline v_d}{E} = - \dfrac{e_0 \overline \tau}{2 m_0}
    \end{equation}
    bezeichnet, wobei E in diesem Fall das Elektrische Feld darstellt.
