\section{Durchführung}
\label{sec:Durchführung}

\subsection{Bestimmung der Fließgeschwindigkeit}
    Untersucht wird eine Flüssigkeitgemisch aus Wasser, Glycerin und Glaskugeln. 
    Die Strömungsgeschwindigkeit wird in drei Rohren verschiedenen Durchmessers (7 mm,
    10 mm, 16 mm) bestimmt. Es werden Dopplerprismen mit drei verschiedenen Einschallwinkeln
    (15°, 30° und 60°) zur Ankopplung der Ultraschallsonde an die Rohre verwendet. 
    Aufgrund von Brechung sind dies allerdings nicht die Winkel, unter denen die Schallwellen
    auf die Glaskugeln treffen. Diese berechnen sich zu
    \begin{equation}
        \alpha = 90° - \arcsin{\sin{\theta}\dfrac{c_L}{c_P}}
    \end{equation}
    mit Schallgeschwindigkeit $c_L$ der Flüssigkeit und $c_P$ des Prismas. Bei den Messungen werden 
    sowohl die Messwinkel als auch die Fließgeschwindigkeit variiert.

\subsection{laminare Strömungen}
    Durch das Glycerin besitzt die Dopplerflüssigkeit eine gewisse Viskosität. Das führt dazu,
    dass laminare Strömungen entstehen, also das Wasser in der Mitte des Schlauches
    schneller fließt, als am Rand. Das Strömungsprofil wird im mitteldicken Schlauch 
    bestimmt. Dazu wird die gleich Technik wie zuvor und ein Winkel von 15° verwendet.
    Variiert wird zwischen den Messungen die Messtiefe.
