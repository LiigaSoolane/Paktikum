\section{Diskussion}
\label{sec:Diskussion}

Die in der Auswertung bestimmten Werte sind in Tabelle \ref{tab:mess}
dargestellt. 

\begin{table}
    \centering
    \begin{tabular}{c c @{${}\pm{}$} c}
        \toprule
        & \multicolumn{2}{c}{Messwert} \\
        \midrule
        $E_{K_\alpha} [\text{eV}]$ & 8044 & 34 \\
        $E_{K_\beta} [\text{eV}]$ & 8910 & 40 \\
        $\lambda_{Compton} [\text{pm}]$ & 3.76 & 0.06\\ 
        \bottomrule
    \end{tabular}
    \caption{Messwerte der Energien und Compton-Wellenlänge.}
    \label{tab:mess}
\end{table}
\noindent Die Theoriewerte der $K_\alpha$ und $K_\beta$ Linien sind in Tabelle
\ref{tab:theo} zu sehen. Daraus kann nach %\eqref{eqn:}
die Compton-Wellenlänge bestimmen. Sie lautet $\lambda_c = 2.427 \si{\pico\m}$.

\begin{table}
    \centering
    \begin{tabular}{c c c c}
        \toprule
        & $E [\text{eV}]$ & $\lambda [\text{m}]$ & $\theta [°]$\\
        \midrule
        $E_{K_{\alpha}}$ & 8048,11 & 15,92 & 21,8\\
        $E_{K_{\beta}} $ & 8906,9 & 13,99 & 19,7\\
        \bottomrule
    \end{tabular}
    \caption{Literaturwerte der Energie, Wellenlänge und Winkel \cite{theo}.}
    \label{tab:theo}
\end{table}
\noindent Zwischen dem Messwert und dem Theoriewert für die $K_\alpha$ Linie liegt ein prozentualer
Unterschied von $0.051 \%$, für die $K_\beta$ Linie liegt er bei $0.09 \%$.
In beiden Fällen sind das sehr geringe Abweichungen, die auf eine genaue Messung
hindeuten. \\
Bei der Compton-Wellenlänge dagegen liegt die prozentuale Abweichung
zwischen Theorie- und Messwert bei $54.9 \%$. Dieser sehr große Unterschied
ist teilweise der sehr kleinen Skala geschuldet: Da der besagte Wert sehr klein ist,
werden kleine Messunsicherheiten zu größeren Fehlern. Die Daten wurden zur Verfügung 
gestellt, daher ist eine Aussage über mögliche Fehler bei Durchführung und Aufbau nicht
möglich.\\
Es handelt sich allerdings um einen statistischen Prozess, daher kommt ein statistischer Fehler 
hinzu, der selbst bei genauer Messung nicht verschwindet. 

\paragraph{Compton-Effekt bei sichtbarem Licht}

Die Wellenlänge von sichtbarem Licht liegt zwischen $380$ und $780 \si{\nano\m}$. Sie ist 
also auf jeden Fall um mehrere Größenordnungen größer als die Compton-Wellenlänge. Es ist daher
nicht möglich, den Compton-Effekt bei Licht aus dem sichtbarem Frequenzbereich zu beobachten. 