\section{Diskussion}
\label{sec:Diskussion}

Zunächst wurde die Charakteristik des Geiger-Müller-Zählrohrs untersucht. Dafür wurde bei Variation
der Zählrohrspannung die Anzahl der Impulse und der Strom gemessen. Dafür wurden folgende Werte 
gemessen:
\begin{table}[H]
    \centering
    \caption{gemessene Impulse in Abhängigkeit der Spannung}
    \label{tab:data}
    \begin{tabular}{c c}
    \toprule
    U [V] & N [Imp] \\
    \midrule
    320.0000 &  9672.0000 \\   
    330.0000 &  9689.0000 \\
    340.0000 &  9580.0000 \\   
    350.0000 &  9837.0000 \\   
    360.0000 &  9886.0000 \\   
    370.0000 & 10041.0000 \\   
    380.0000 &  9996.0000 \\   
    390.0000 &  9943.0000 \\   
    400.0000 &  9995.0000 \\   
    410.0000 &  9980.0000 \\   
    420.0000 &  9986.0000 \\   
    430.0000 &  9960.0000 \\   
    440.0000 & 10219.0000 \\   
    450.0000 & 10264.0000 \\   
    460.0000 & 10174.0000 \\   
    470.0000 & 10035.0000 \\   
    480.0000 & 10350.0000 \\   
    490.0000 & 10290.0000 \\   
    500.0000 & 10151.0000 \\   
    510.0000 & 10110.0000 \\   
    520.0000 & 10255.0000 \\   
    530.0000 & 10151.0000 \\   
    540.0000 & 10351.0000 \\   
    550.0000 & 10184.0000 \\   
    560.0000 & 10137.0000 \\   
    570.0000 & 10186.0000 \\   
    580.0000 & 10171.0000 \\   
    590.0000 & 10171.0000 \\   
    600.0000 & 10253.0000 \\   
    610.0000 & 10368.0000 \\   
    620.0000 & 10365.0000 \\   
    630.0000 & 10224.0000 \\   
    640.0000 & 10338.0000 \\   
    650.0000 & 10493.0000 \\   
    660.0000 & 10467.0000 \\   
    670.0000 & 10640.0000 \\
    680.0000 & 10939.0000 \\   
    690.0000 & 11159.0000 \\
    700.0000 & 11547.0000 \\ 
    \bottomrule
  \end{tabular}
\end{table}
\noindent Die daraus gewonnenen Werte für die Länge $L= 270 V$ und die Steigung $a = 1.262 \pm 0.218$ \% 
pro 100 V sind nicht ideal, aber für die folgende Messung ausreichend gut. \\
Die Bestimmung der Totzeit anhand des Oszilloskops konnte leider nur anhand eines Bildes 
geringer Größe bestimmt werden, sodass hier eine hohe Messunsicherheit besteht.\\
Die Totzeitbestimmung durch die Zwei-Quellen-Methode ist ebenfalls nur bedingt aussagekräftig,
da durch die hohe Messunsicherheit von $\sqrt{N}$ des Wert N auch hier Unsicherheiten bestehen.
Da aber die durch das Oszilloskop bestimmte Totzeit innerhalb der Messunsicherheit der Zwei-Quellen-
Methode liegt, und somit ein ählicher Wert aus zwei unterschiedlichen Messmethoden gewonnen 
wurde, gehen wir davon aus, dass die Berechnung der Totzeit Erfolg hatte.\\
Auch die Bestimmung der pro Teilchen freigesetzte Ladung ist wegen der hohen Messunsicherheit von 
N mit Vorsicht zu genießen.
