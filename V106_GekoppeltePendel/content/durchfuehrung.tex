\section{Durchführung}
\label{sec:Durchführung}

\subsection{Versuchsaufbau}
    Für den Versuch werden zwei Stabpendel mit einer reibungsarmen Spitzenlagerung verwendet. Durch verschiebbare
    Pendelkörper können verschiedene Pendellängen eingestellt werden. Die nachfolgenden 
    Messungen werden für zwei verschiedene Pendellängen durchgeführt. Die Höhe der Feder, 
    welche die Pendel koppelt, wird nicht variiert.
\subsection{Versuchsdurchführung}
    Zunächst werden die Pendel entkoppelt, um anhand der Periodendauern festzustellen,
    ob die beiden Stabpendel die gleiche Pendellänge besitzen. Hierfür werden jeweils 5 
    Schwingungsdauern gemessen. Damit der menschliche Fehler (ungenaues Drücken der Stoppuhr) 
    möglichst gering gehalten werden kann, wird die Messung 10 mal durchgeführt und dann 
    werden die Mittelwerte verglichen. Liegt $(\overline T_1-\overline T_2)$ innerhalb der 
    Standartabweichung der beiden Pendel, ist die Abweichung der Pendellängen vernachlässigbar 
    und die Längen können verwendet werden. \\
    Nun werden die Pendel durch die Feder gekoppelt und es werden nacheinander durch 
    geeignetes Auslenken die beiden Eigenmoden angeregt. Zwecks Fehlerminimierung wird 
    erneut für 5 Schwingungsdauern gemessen. Auch hier werden jeweils 10 Messungen durchgeführt.\\
    Zu guter Letzt wird eine gekoppelte Schwingung erzeugt und die Schwingungsdauer $T$ und 
    die Schwebungsdauer $T_S$ gemessen. Wie bei den vorangegangenen Messungen werden 10 Messungen a 5 
    Schwingungen durchgeführt.